% Created 2015-01-31 Sat 19:17
\documentclass[11pt]{article}
\usepackage[utf8]{inputenc}
\usepackage[T1]{fontenc}
\usepackage{fixltx2e}
\usepackage{graphicx}
\usepackage{longtable}
\usepackage{float}
\usepackage{wrapfig}
\usepackage{rotating}
\usepackage[normalem]{ulem}
\usepackage{amsmath}
\usepackage{textcomp}
\usepackage{marvosym}
\usepackage{wasysym}
\usepackage{amssymb}
\usepackage{hyperref}
\tolerance=1000
\usepackage[utf8]{inputenc}
\usepackage[usenames,dvipsnames]{color}
\usepackage[backend=bibtex, style=numeric]{biblatex}
\usepackage{commath}
\usepackage{tikz}
\usetikzlibrary{shapes,backgrounds}
\usepackage{marginnote}
\usepackage{listings}
\usepackage{color}
\usepackage{enumerate}
\hypersetup{urlcolor=blue}
\hypersetup{colorlinks,urlcolor=blue}
\addbibresource{bibliography.bib}
\setlength{\parskip}{16pt plus 2pt minus 2pt}
\definecolor{codebg}{rgb}{0.96,0.99,0.8}
\definecolor{codestr}{rgb}{0.46,0.09,0.2}
\author{Oleg Sivokon}
\date{\textit{<2015-01-30 Fri>}}
\title{Assignment 17, Introduction To Mathematics}
\hypersetup{
  pdfkeywords={Introduction To Mathematics, Assignment, Set Theory},
  pdfsubject={Seventh asssignment in the course Introduction To Mathematics},
  pdfcreator={Emacs 25.0.50.1 (Org mode 8.2.2)}}
\begin{document}

\maketitle
\tableofcontents


\lstset{ %
  backgroundcolor=\color{codebg},
  basicstyle=\ttfamily\scriptsize,
  breakatwhitespace=false,         % sets if automatic breaks should only happen at whitespace
  breaklines=false,
  captionpos=b,                    % sets the caption-position to bottom
  commentstyle=\color{mygreen},    % comment style
  framexleftmargin=10pt,
  xleftmargin=10pt,
  framerule=0pt,
  frame=tb,                        % adds a frame around the code
  keepspaces=true,                 % keeps spaces in text, useful for keeping indentation of code (possibly needs columns=flexible)
  keywordstyle=\color{blue},       % keyword style
  showspaces=false,                % show spaces everywhere adding particular underscores; it overrides 'showstringspaces'
  showstringspaces=false,          % underline spaces within strings only
  showtabs=false,                  % show tabs within strings adding particular underscores
  stringstyle=\color{codestr},     % string literal style
  tabsize=2,                       % sets default tabsize to 2 spaces
}

\clearpage

\section{Problems}
\label{sec-1}

\subsection{Problem 1}
\label{sec-1-1}
\begin{enumerate}
\item Let $A$ and $B$ be points, and let $\ell$ be a line crossing the segment
$AB$ in the point $C$.  Prove that from four axioms of incidence and the
Pasch's axiom it follows that there are two distinct points on $\ell$.
\item Prove from four axioms of incidence and the axioms of order it follows
that there doesn't exist a line which contains just one point.
\end{enumerate}

\subsubsection{Answer 1}
\label{sec-1-1-1}
Sorry, I don't know what definitions do you have in mind, but every
definition of incidence geometry that I could find, for example this one:
\url{https://www.imsc.res.in/~kapil/geometry/euclid/node2.html} flat out says
\emph{Every line contains at least two distinct points.} This means that the
proof is extremely trivial, it is simply the reiteration of the first
axiom.

Anoter reference: \href{http://www.uwyo.edu/moorhouse/handouts/incidence_geometry.pdf}{Moorhouse, G. Eric. "Incidence Geometry"} (page 5):

\emph{The preceding paragraph talks about partial linear space:}

\begin{quote}
(PLS1) Any two distinct points lie on at most one common line.
(PLS2) Every line has at least two points.
\end{quote}

\emph{And even stricter for linear space:}
\begin{quote}
(LS1) Any two distinct points lie on exactly one common line.
(LS2) Every line has at least two points.
\end{quote}
\subsubsection{Answer 2}
\label{sec-1-1-2}
Exactly by the same reasoning as \ref{sec-1-1-1} the proof is trivially the
reiteration of the first axiom.  I will also note that whether we use
order axioms or just Pasch's axiom is inconsequential for the proof.
\subsection{Problem 2}
\label{sec-1-2}
\begin{enumerate}
\item Prove using the four axioms of incidence and the axiom of parallels 
that there exist at least six lines.
\item Does it follow from the same system as mentioned in (1) that there
exist seven lines?
\end{enumerate}

\subsubsection{Answer 3}
\label{sec-1-2-1}
In order to prevent confusion I will list the axioms in the way that
I'm familiar with:

\begin{enumerate}
\item There are at least two distinct points.
\item There is one and only one line that contains two distinct points.
\item Every line contains at least two distinct points.
\item There are three points that do not all lie on the same line.
\item For any three points that do not lie on the same line there is 
a one and only one plane that contains them.
\item Any plane contains at least three points.
\item If a line lies on a plane then every point contained in the line 
lies on that plane.
\item If a line contains two points which lie on a plane then the line 
lies on the plane.
\item If two planes both contain a point then they also contain a line.
\item There are at least four points that do not all lie on the same plane.
\end{enumerate}

The axiom of parallels:

\begin{itemize}
\item Given a line and a point outside it there is exactly one line through the
given point which lies in the plane of the given line and point so that
the two lines do not meet.
\end{itemize}

Without giving a proof, I will note that the smallest model for this
system has 12 planes and 36 lines, so this is probably a wrong system.

Another system of incidence axioms with parallelism I could find was
in Shafarevich's Basic Notions of Algebras:

\begin{enumerate}
\item Through any two distinct points there is one and only one line.
\item Given any line and a point not on it, there exists one and only one
other line through the point and not intersecting the line (that is,
parallel to it).
\item There exist three points not on any line.
\end{enumerate}

The minimal (in terms of objects used) model for this system may be
a ``tick-tack-toe'' with two extra ``diagonal'' lines crossing it, in
other words, it is the points $A, B, C, D$ and lines $AB$, $BC$, $CD$,
$DA$, $AC$ and $BD$.  The following is an attempt to prove that such
system is minimal.

From (3) follows there are at least 3 points, let's name them $A$,
$B$ and $C$.  By (1) this gives us that there must be these lines:
$AB$, $BC$ and $CA$.  In order to satisfy (2) we need add one more point.
This is so because we know that $A$, $B$ and $C$ are not on the same
line.  Observe now that any of the three points is not on the same
line with any two remaing points.  But, by adding point $D$ we are
able to satisfy the requirement of (2) by adding three more lines:
$CD$, $DA$ and $BD$.  Now all requirements are satisfied and we have
exactly six lines in the model.  Hence, there must be at least six
lines.
\subsubsection{Answer 4}
\label{sec-1-2-2}
As can be seen from the \ref{sec-1-2-1}, there is no requirement that there
be seven lines in the system (I just demonstrated a model with
exactly six lines).
\subsection{Problem 3}
\label{sec-1-3}
Prove or disprove:
\begin{enumerate}
\item If $n$ is a natural number, then if $6|n^2$ $6|n$.
\item If $n$ is a natural number, then if $12|n^2$ $12|n$.
\item Let $A*$ be a set generated by multiplication from 
      $\{24, \frac{1}{8}, \frac{1}{3}\}$ then ${288 \in A*}$.
\item There exist natural numbers $x,y,z,t$ such that \\
      ${28^x \times 21^{y-1} -16^z \times 49^t = 0}$.
\end{enumerate}

\subsubsection{Answer 5}
\label{sec-1-3-1}
Yes, this follows from uniqueness of prime factorization.  Any square
of a real number will have to contain at least twice every one of the
prime factors of $n$.  This, in turn, implies that if $n^2$ is divisible
by 6, then it has to have both 3 and 2 twice in its prime factorization,
therefore $\sqrt{n^2}$ will contain both 2 and 3, therefore it will be
divisible by 6.
\subsubsection{Answer 6}
\label{sec-1-3-2}
This is not necessarily so since if $n^2$ is divisible by 12, it needs
not have the prime factor 2 repeated four times (two would be still enough).
So, for example $n=6$, $6\times6=36$, 36 is divisible by 12, but 6 isn't.
\subsubsection{Answer 7}
\label{sec-1-3-3}
No, this isn't possible because of the prime factorization of 288 containing
five twos, but it is not possible to produce this many twos from prime
factorizations of 24 and 8.  In other words any multiple of 8 and 24 will
have a number of twos divisible by three, and 3 (the remaining member of
the generating set) has no influence on the number of twos in the generated
set members.
\subsubsection{Answer 8}
\label{sec-1-3-4}
Yes, this is possible.  If we look at the prime factorization of the
left product, it has $\{2, 7, 3\}$, and the product on the right has
prime factorization containing $\{2, 7\}$.  But we can eliminate the
factor 3 from the first factorization by setting $y=1$, thus removing
it from equation.  And, indeed, assignment $x=2, y=1, z=1, t=1$ solves
this equation.
\subsection{Problem 4}
\label{sec-1-4}
\begin{enumerate}
\item Given a sequence defined as follows:
$a_1=1$, and for all $n$ being natural numbers,
$a_{n+1}=a_n+\frac{1}{n(n+1)}$.  Prove by induction that for all $n$
being natural numbers $a_n = 2 - \frac{1}{n}$.
\item Prove using induction that for all $n$ being natural numbers it holds
that $13|10^{2n-1}+3^{2n-1}$.
\end{enumerate}

\subsubsection{Answer 9}
\label{sec-1-4-1}
The conjecture is false, already for the value of $n=2$ both formulae give
different values:

$a_n+\frac{1}{n(n+1)}=1+\frac{1}{2(2+1)}=\frac{7}{6}$

while:

$2 - \frac{1}{n}=2 - \frac{1}{2}= \frac{3}{2}$.
\subsubsection{Answer 10}
\label{sec-1-4-2}
I will prove, using mathematical induction on $n \in \mathbb{N}$ that 13
divides any sum of odd powers of 10 and 3.  First, the intuition for the
proof: what we need to convince ourselves in is that the number added at any
recursion step must be divisible by 13.  This is so since if $x$ is
divisible by $y$, then if $x+z$ is divisible by $y$, then it follows that
$z$ is, too, divisible by $y$.  The most work of this proof is centered
around finding $z$.

\textbf{Base step} is trivially true: $10+3=13$, $13|13$.

\textbf{Induction step.} I will reformulate the original statement equivalently
as $10^{2n-1}+3^{2n-1}=13x$ where $x \in \mathbb{N}$.  This will
simplify the naming.

\begin{align*}
  10^{2n-1}+3^{2n-1} &= 13x \\
  10^{2(n+1)-1}+3^{2(n+1)-1} &= 13x + 10^{2n-1}+3^{2n-1} \\
  & \textrm{by induction hypothesis} \\
  10^{2n+1}+3^{2n+1} &= 13x + 10^{2n-1}+3^{2n-1} \\
  & \textrm{by simple calculus} \\
  10^{2n+1}+3^{2n+1}-10^{2n-1}-3^{2n-1} &= 13x \\
  & \textrm{collecting summands on one side} \\
  10^{2n-1}(10^2)-10^{2n-1}+3^{2n-1}(3^2)-3^{2n-1} &= 13x \\
  & \textrm{by $x^{y+z}=x^yx^z$} \\
  10^{2n-1}(99)+3^{2n-1}(8) &= 13x \\
  & \textrm{by simple calculus} \\
  10^{2n-1}(91) &= 13x' \\
  & \textrm{$13|10^{2n-1}(8)+3^{2n-1}(8)$} \\
  & \textrm{by induction hypothesis} \\
  10^{2n-1}(13*7) &= 13x' \\
  & \textrm{$x' \in \mathbb{N}$} \\
\end{align*}

In other words, I've showed that the number added at each recursive step must
be divisible by 13, hence, by mathematical induction, $13|10^{2n-1}+3^{2n-1}$.
% Emacs 25.0.50.1 (Org mode 8.2.2)
\end{document}