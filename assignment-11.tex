% Created 2014-10-17 Fri 17:30
\documentclass[11pt]{article}
\usepackage[utf8]{inputenc}
\usepackage[T1]{fontenc}
\usepackage{fixltx2e}
\usepackage{graphicx}
\usepackage{longtable}
\usepackage{float}
\usepackage{wrapfig}
\usepackage{rotating}
\usepackage[normalem]{ulem}
\usepackage{amsmath}
\usepackage{textcomp}
\usepackage{marvosym}
\usepackage{wasysym}
\usepackage{amssymb}
\usepackage{hyperref}
\tolerance=1000
\usepackage[utf8]{inputenc}
\usepackage[usenames,dvipsnames]{color}
\usepackage[backend=bibtex, style=numeric]{biblatex}
\usepackage{commath}
\usepackage{tikz}
\usetikzlibrary{shapes,backgrounds}
\usepackage{marginnote}
\usepackage{minted}
\usemintedstyle{perldoc}
\hypersetup{urlcolor=blue}
\hypersetup{colorlinks,urlcolor=blue}
\addbibresource{bibliography.bib}
\setlength{\parskip}{16pt plus 2pt minus 2pt}
\definecolor{codebg}{rgb}{0.96,0.99,0.8}
\author{Oleg Sivokon}
\date{2014-10-16}
\title{Assignment 11, Introduction To Mathematics}
\hypersetup{
  pdfkeywords={Introduction To Mathematics, Assignment, Set Theory},
  pdfsubject={First asssignment in the course Introduction To Mathematics},
  pdfcreator={Emacs 24.3.50.1 (Org mode 8.2.2)}}
\begin{document}

\maketitle
\tableofcontents


\clearpage

\section{Problems}
\label{sec-1}

\subsection{Problem 1}
\label{sec-1-1}

Let $A$ and $B$ be defined as follows:

\begin{equation}
\begin{split}
& A = \{1, 4, 9, ...\} = \{n^2 | n \in \mathbb{N}\} \\
& B = \{1, 16, 81, ...\} = \{n^4 | n \in \mathbb{N}\}
\end{split}
\end{equation}

\begin{enumerate}
\item Find a bijection from $A$ to $B$.
\item Find a non-bijective binary relation $A R B$.
\item What can you conclude about equivalence between $A$ and $B$,
based on questions 1 and 2?
\item Do previous questions imply that $A$ is infinite?
\end{enumerate}

\subsubsection{Answer 1}
\label{sec-1-1-1}
There must be many ways to find a bijection between $A$ and $B$,
but I will proceed by finding $f:A \rightarrow \mathbb{N}$ and
$g:B \rightarrow \mathbb{N}$.  Provided I can find these, than it
must be the case that I found a bijection from $A$ to $B$.

I will define $f$ and $g$ as follows:

\begin{equation}
\begin{split}
& f(x) = \norm{\sqrt[2]{x}} \\
& g(x) = \norm{\sqrt[4]{x}}
\end{split}
\end{equation}

Clearly, it is the case that
$\forall x, y \in A. x \neq y \implies f(x) \neq f(y)$, similarly
$\forall x, y \in B. x \neq y \implies f(x) \neq f(y)$.  In addition,
we are guranteed that $f(A) \in \mathbb{N}$ and $g(B) \in \mathbb{N}$
by construction.  Dissimilarity of this kind is the defining property of
bijection.

Since $f(A)$ is equivalent to $\mathbb{N}$ and $g(B)$ is equivalent to
$\mathbb{N}$, it follows that they are equivalent under $f \circ g$.
Not only that, we had actually proved a stronger claim that
$f(A) = g(B) = \mathbb{N}$.
\subsubsection{Answer 2}
\label{sec-1-1-2}
One non-bijective relation $R$ on $A$ and $B$ can be defined as follows:

\begin{equation}
R(A, B) = \{(a, b) | b \in A \land b \in B \land a = b \}
\end{equation}

In other words, this relation will only select such elements of $A$, which
are also present in $B$.  Since $4 \not \in B$, there is no element matched
to it in $R$.  Consequently, $R$ is not a bijection.
\subsubsection{Answer 3}
\label{sec-1-1-3}
There are many different equivalence relations that can be established
between $A$ and $B$, an example different from the one used in \ref{sec-1-1-1}
could be:
\begin{equation}
\{(a, b) | a \in A \land b \in B \land
((a \bmod 2 = 1 \land (a - 1)^2 = b) \lor (a + 1)^2 = b)\}
\end{equation}

Such equivalence relation would select all odd members of $A$ and send them
to the member of $B$ such as it would be the square of it plus one.  It
would do the reverse for even numbers (effectively swapping between the
two).  Yet, all of these equivalence relations would have to be within the
$\aleph_0$ equivalence class.
\subsubsection{Answer 4}
\label{sec-1-1-4}
$A$ and $B$ are infinite if only for the reason that they are equivalent to
$\mathbb{N}$, which is known to be infinite, but this can be seen without
reliance on equivalence relation simply by observing that we can always
construct a member of $A$ or $B$ larger than any of the members constructed
so far.
\subsection{Problem 2}
\label{sec-1-2}
The Venn diagram below describes the relation between sets $A, B, C$ and $E$.
Paint the areas according to the expressions given below.

\begin{enumerate}
\item $B \setminus (A \setminus C)$
\item $B \setminus (C \setminus A)$
\item $((A \setminus B) \setminus C) \cup (A \cap B^c(E) \cup C^c(E))$
\item $(A \cup B)^c(E) \cup (B \cup C)^c(E) \cup (C \cup A)^c(E)$
\item $(A \cup B) \cap (B \cup C) \cap (C \cup A)$
\end{enumerate}

\def\firstcircle{(2,2.8) circle (1.2cm)}
\def\secondcircle{(4,2.8) circle (1.2cm)}
\def\thirdcircle{(3,1.4) circle (1.2cm)}
\def\background{(0,0) rectangle (6,4.2cm)}

\begin{tikzpicture}
  \draw \background node [above] {$E$};
  \draw \firstcircle node {$A$};
  \draw \secondcircle node {$B$};
  \draw \thirdcircle node {$C$};
\end{tikzpicture}

\subsubsection{Answer 5}
\label{sec-1-2-1}
$B \setminus (A \setminus C)$

\begin{tikzpicture}
  \begin{scope}[shift={(6cm,0cm)}]
    \begin{scope}
      \clip \thirdcircle;
      \clip \firstcircle;
      \fill[yellow] \secondcircle;
    \end{scope}
    \begin{scope}[even odd rule]
      \clip \firstcircle (0,0) rectangle (6,6);
      \fill[yellow] \secondcircle;
    \end{scope}
    \draw \background node [above] {$E$};
    \draw \firstcircle node {$A$};
    \draw \secondcircle node {$B$};
    \draw \thirdcircle node {$C$};
  \end{scope}
\end{tikzpicture}
\subsubsection{Answer 6}
\label{sec-1-2-2}
$B \setminus (C \setminus A)$

\begin{tikzpicture}
  \begin{scope}[shift={(6cm,0cm)}]
    \begin{scope}
      \clip \thirdcircle;
      \clip \firstcircle;
      \fill[yellow] \secondcircle;
    \end{scope}
    \begin{scope}[even odd rule]
      \clip \thirdcircle (0,0) rectangle (6,6);
      \fill[yellow] \secondcircle;
    \end{scope}
    \draw \background node [above] {$E$};
    \draw \firstcircle node {$A$};
    \draw \secondcircle node {$B$};
    \draw \thirdcircle node {$C$};
  \end{scope}
\end{tikzpicture}
\subsubsection{Answer 7}
\label{sec-1-2-3}
$((A \setminus B) \setminus C) \cup (A \cap B^c(E) \cup C^c(E))$

\begin{tikzpicture}
  \begin{scope}[shift={(6cm,0cm)}]
    \begin{scope}
      \clip \thirdcircle (0,0) rectangle (6,6);
      \clip \secondcircle (0,0) rectangle (6,6);
      \fill[yellow] \firstcircle;
    \end{scope}
    \begin{scope}[even odd rule]
      \clip \thirdcircle (0,0) rectangle (6,6);
      \fill[yellow] \firstcircle;
    \end{scope}
    \begin{scope}[even odd rule]
      \clip \secondcircle (0,0) rectangle (6,6);
      \fill[yellow] \firstcircle;
    \end{scope}
    \draw \background node [above] {$E$};
    \draw \firstcircle node {$A$};
    \draw \secondcircle node {$B$};
    \draw \thirdcircle node {$C$};
  \end{scope}
\end{tikzpicture}
\subsubsection{Answer 8}
\label{sec-1-2-4}
$(A \cup B)^c(E) \cup (B \cup C)^c(E) \cup (C \cup A)^c(E)$

\begin{tikzpicture}
  \begin{scope}[shift={(6cm,0cm)}]
    \begin{scope}
      \clip \firstcircle (0,0) rectangle (6,6);
      \clip \secondcircle (0,0) rectangle (6,6);
      \clip \thirdcircle (0,0) rectangle (6,6);
      \fill[yellow] \background;
    \end{scope}
    \begin{scope}
      \clip \secondcircle (0,0) rectangle (6,6);
      \fill[yellow] \firstcircle;
    \end{scope}
    \begin{scope}
      \clip \thirdcircle (0,0) rectangle (6,6);
      \fill[yellow] \secondcircle;
    \end{scope}
    \begin{scope}
      \clip \firstcircle (0,0) rectangle (6,6);
      \fill[yellow] \thirdcircle;
    \end{scope}
    \draw \background node [above] {$E$};
    \draw \firstcircle node {$A$};
    \draw \secondcircle node {$B$};
    \draw \thirdcircle node {$C$};
  \end{scope}
\end{tikzpicture}
\subsubsection{Answer 8}
\label{sec-1-2-5}
$(A \cup B) \cap (B \cup C) \cap (C \cup A)$

\begin{tikzpicture}
  \begin{scope}[shift={(6cm,0cm)}]
    \begin{scope}
      \clip \secondcircle;
      \fill[yellow] \firstcircle;
    \end{scope}
    \begin{scope}
      \clip \thirdcircle;
      \fill[yellow] \secondcircle;
    \end{scope}
    \begin{scope}
      \clip \firstcircle;
      \fill[yellow] \thirdcircle;
    \end{scope}
    \draw \background node [above] {$E$};
    \draw \firstcircle node {$A$};
    \draw \secondcircle node {$B$};
    \draw \thirdcircle node {$C$};
  \end{scope}
\end{tikzpicture}
\subsection{Problem 3}
\label{sec-1-3}
Given sets $A$ and $B$, such that $\forall x \in B. A \setminus \{x\} \equiv A$,
prove or disprove:

\begin{enumerate}
\item If $A$ is finite, then $A \cap B = \emptyset$.
\item If $B$ is finite, then $A \cap B = \emptyset$.
\end{enumerate}

\subsubsection{Answer 9}
\label{sec-1-3-1}
If $A$ is finite, then $A \cap B = \emptyset$ holds, because if it wasn't the
case, then there had to be $x \in B$ such that it would be both a member of $A$
and of $B$.  If we then pair up every element of $A$ to $A \setminus \{x\}$ using,
identity relation (which is known to be bijective), that particular $x$ would had
no pair.  The later would contradict the initial assumption
$A \equiv A \setminus \{x\}$.
\subsubsection{Answer 10}
\label{sec-1-3-2}
If $B$ is finite, the statement $A \cap B = \emptyset$ doesn't always hold.
One such example can be given by assignment:

\begin{equation}
\begin{split}
& A = \mathbb{N} \\
& B = \{1\} \\
& R(A, A \setminus B) = \{(x, x + 1) | x \in A\}
\end{split}
\end{equation}

Where $R$ is a binary bijective relation on $A$ and $A \setminus B$.
\subsection{Problem 4}
\label{sec-1-4}
Given sets $A$ and $B$ prove or disprove:

\begin{enumerate}
\item $(A = A \setminus B) \implies (B = \emptyset)$.
\item $(A = A \setminus B) \implies (A \cap B = \emptyset)$.
\item $(A \equiv A \setminus B) \implies (A \cap B = \emptyset)$.
\item $(Finite(A) \land (A \equiv A \setminus B)) \implies (A \cap B = \emptyset)$.
\end{enumerate}

\subsubsection{Answer 11}
\label{sec-1-4-1}
$(A = A \setminus B) \implies (B = \emptyset)$ doesn't hold because it is
possible for all elements of $B$, however many, not to be elements of $A$.
To illustrate the contradiction we construct this example:

\begin{equation}
\begin{split}
& A = \{1\} \\
& B = \{2\} \\
& A \setminus B = \{1\} = A
\end{split}
\end{equation}
\subsubsection{Answer 12}
\label{sec-1-4-2}
$(A = A \setminus B) \implies (A \cap B = \emptyset)$ holds, because if
it wasn't the case, then there had to be such $b \in B$, which is also the
member of $A$.  Subtracting $b$ from $A$ would thus had created a set
distinct from $A$, but we are given $A = A \setminus \{b\}$.  Hence, this
is impossible.  Hence, the original claim holds.
\subsubsection{Answer 13}
\label{sec-1-4-3}
This is the exact replica of the question \ref{sec-1-3-2}. The statement 
$(A \equiv A \setminus B) \implies (A \cap B = \emptyset)$ doesn't hold for
infinite $A$.
\subsubsection{Answer 14}
\label{sec-1-4-4}
As this is a refinement of question \ref{sec-1-4-3}, and as I mentioned earlier,
This statement is indeed true.  Finite sets can only be equivalent if they
are also equal.  Subracting an element from a finite set creates a distinct
set.  Thus it must be the case that we did not subtract any elements from
$A$, but that would be only possible if $A \cap B = \emptyset$.
\section{Exercises}
\label{sec-2}
Given $A$, $B$ and $C$ are sets, $\emptyset$ is the empty set and $\mathbb{N}$
is the set of natural numbers, $Finite(X)$ and $Infinite(X)$ are predicates
which are true if $X$ is finite or infinite, $R$ is a binary relation and
$Bijection(X)$ is true when relation $X$ is a bijection, answer:

\begin{itemize}
\item \textbf{a} if only the first statement is correct.
\item \textbf{b} if only the second statement is correct.
\item \textbf{c} if both statements are correct.
\item \textbf{d} if neither statement is correct.
\end{itemize}

\subsection{Exercise 1}
\label{sec-2-1}
\begin{equation}
\begin{split}
& 1.\; \{1, 2\} \subseteq \{1, \{1, 2\}\} \\
& 2.\; \{1, 2\} \in \{1, \{1, 2\}\}
\end{split}
\end{equation}

\emph{Answer:} \textbf{b}
\subsection{Exercise 2}
\label{sec-2-2}
\begin{equation}
\begin{split}
& 1.\; \{1\} \subseteq \{1, \{1, 2\}\} \\
& 2.\; \{1\} \in \{1, \{1, 2\}\}
\end{split}
\end{equation}

\emph{Answer:} \textbf{a}
\subsection{Exercise 3}
\label{sec-2-3}
\begin{equation}
\begin{split}
& 1.\; \emptyset \subseteq \{1, 2\} \\
& 2.\; \emptyset \in \{1, 2\}
\end{split}
\end{equation}

\emph{Answer:} \textbf{a}
\subsection{Exercise 4}
\label{sec-2-4}
\begin{equation}
\begin{split}
& 1.\; (A \subset B) \implies (A \subseteq B) \\
& 2.\; (A \subset B) \implies B \neq \emptyset
\end{split}
\end{equation}

\emph{Answer:} \textbf{c}
\subsection{Exercise 5}
\label{sec-2-5}
\begin{equation}
\begin{split}
& 1.\; (x \not \in A) \implies (x \not \in (A \cap B)) \\
& 2.\; (x \not \in A) \implies (x \not \in (A \cup B))
\end{split}
\end{equation}

\emph{Answer:} \textbf{b}
\subsection{Exercise 6}
\label{sec-2-6}
\begin{equation}
\begin{split}
& 1.\; (x \not \in (A \cup B)) \implies ((x \not \in A) \land (x \not \in B)) \\
& 2.\; (x \not \in (A \cap B)) \implies ((x \not \in A) \land (x \not \in B))
\end{split}
\end{equation}

\emph{Answer:} \textbf{a}
\subsection{Exercise 7}
\label{sec-2-7}
\begin{equation}
\begin{split}
& 1.\; (x \in (A \setminus B)) \implies (x \not \in B) \\
& 2.\; (x \not \in (A \setminus B)) \implies ((x \not \in A) \lor (x \in B))
\end{split}
\end{equation}

\emph{Answer:} \textbf{c}
\subsection{Exercise 8}
\label{sec-2-8}
\begin{equation}
\begin{split}
& 1.\; (A \not \subseteq B) \implies (A \cap B = \emptyset) \\
& 2.\; (A \subseteq B) \implies (A \cap B \neq \emptyset)
\end{split}
\end{equation}

\emph{Answer:} \textbf{b}
\subsection{Exercise 9}
\label{sec-2-9}
\begin{equation}
\begin{split}
& 1.\; (A \not \subseteq B) \implies (A \cap B = \emptyset) \\
& 2.\; (A \subseteq B) \implies (A \cap B \neq \emptyset)
\end{split}
\end{equation}

\emph{Answer:} \textbf{b}
\subsection{Exercise 10}
\label{sec-2-10}

\begin{tikzpicture}
  \begin{scope}[shift={(6cm,0cm)}]
    \begin{scope}
      \clip \thirdcircle;
      \fill[yellow] \secondcircle;
    \end{scope}
    \begin{scope}[even odd rule]
      \clip \thirdcircle (0,0) rectangle (6,6);
      \fill[yellow] \firstcircle;
    \end{scope}
    \draw \background node [above] {$E$};
    \draw \firstcircle node {$A$};
    \draw \secondcircle node {$B$};
    \draw \thirdcircle node {$C$};
  \end{scope}
\end{tikzpicture}

The area painted in the diagram is accurately described by:

\begin{equation}
\begin{split}
& 1.\; ((A \cup B) \cap C) \cup (A \setminus B) \\
& 2.\; ((A \cup C) \setminus (B \setminus C)) \setminus ((C \setminus B) \setminus A)
\end{split}
\end{equation}

\emph{Answer:} \textbf{d}

\subsection{Exercise 11}
\label{sec-2-11}
\begin{equation}
\begin{split}
& 1.\; (A \neq B) \implies ((A \not \subseteq B) \land (B \not \subseteq A)) \\
& 2.\; (B = B \cup A) \implies (A = A \cap B)
\end{split}
\end{equation}

\emph{Answer:} \textbf{b}
\subsection{Exercise 12}
\label{sec-2-12}
\begin{equation}
\begin{split}
& 1.\; \{1, 2\} \subseteq \{\mathbb{N}\} \\
& 2.\; \{1\} \subseteq \{\mathbb{N}\}
\end{split}
\end{equation}

\emph{Answer:} \textbf{d}
\subsection{Exercise 13}
\label{sec-2-13}
\begin{equation}
\begin{split}
& 1.\; \exists A. A \equiv \{A\} \\
& 2.\; (B \in A) \implies (B \not \equiv A)
\end{split}
\end{equation}

\emph{Answer:} \textbf{a}
\subsection{Exercise 14}
\label{sec-2-14}
\begin{equation}
\begin{split}
& 1.\; Infinite(A) \implies (\forall B. ((B \subset A) \implies (A \equiv B))) \\
& 2.\; Infinite(A) \implies (\exists B. ((B \subseteq A) \implies (A \not \equiv B)))
\end{split}
\end{equation}

\emph{Answer:} \textbf{b}
\subsection{Exercise 15}
\label{sec-2-15}
\begin{equation}
\begin{split}
& 1.\; Infinite(B) \implies (\forall B. ((B \in A) \implies Infinite(A))) \\
& 2.\; Finite(A) \implies (\forall B. ((B \subset A) \implies (B \not \equiv A)))
\end{split}
\end{equation}

\emph{Answer:} \textbf{b}
\subsection{Exercise 16}
\label{sec-2-16}
\begin{equation}
\begin{split}
& 1.\; (B \equiv A) \implies (\forall R. Bijection(R(A, B))) \\
& 2.\; ((A \subset B) \land (A \not \equiv B)) \implies Infinite(B)
\end{split}
\end{equation}

\emph{Answer:} \textbf{d}
\subsection{Exercise 17}
\label{sec-2-17}
\begin{equation}
\begin{split}
& 1.\; (((A \cup B) \equiv A) \land (B \subset A)) \implies Infinite(A \cup B) \\
& 2.\; ((A \cap B) \equiv A) \implies Infinite(A)
\end{split}
\end{equation}

\emph{Answer:} \textbf{d}
\subsection{Exercise 18}
\label{sec-2-18}
\begin{equation}
\begin{split}
& 1.\; Infinite(\{1, \mathbb{N}\}) \\
& 2.\; Infinite(P(\mathbb{N}) \setminus \mathbb{N})
\end{split}
\end{equation}

\emph{Answer:} \textbf{b}
\subsection{Exercise 19}
\label{sec-2-19}
\begin{equation}
\begin{split}
& 1.\; \forall A. (P(A) \neq \emptyset) \\
& 2.\; \forall x. ((x \in A) \implies (x \in P(A)))
\end{split}
\end{equation}

\emph{Answer:} \textbf{d}
\subsection{Exercise 20}
\label{sec-2-20}
\begin{equation}
\begin{split}
& 1.\; ((B \in P(A)) \land (C \subseteq B)) \implies (C \in P(A)) \\
& 2.\; (B \in P(A)) \implies (B \not \in A)
\end{split}
\end{equation}

\emph{Answer:} \textbf{a}
\subsection{Exercise 21}
\label{sec-2-21}
\begin{equation}
\begin{split}
& 1.\; Infinite(A) \implies (A \equiv \mathbb{N}) \\
& 2.\; Infinite(A) \implies (\forall B. ((Infinite(B) \land (B \subseteq A)) \implies (A \equiv B)))
\end{split}
\end{equation}

\emph{Answer:} \textbf{d}
\subsection{Exercise 22}
\label{sec-2-22}
\begin{equation}
\begin{split}
& 1.\; (B \equiv P(A)) \implies (A \not \equiv B) \\
& 2.\; Infinite(A) \implies (P(A) \equiv P(P(A)))
\end{split}
\end{equation}

\emph{Answer:} \textbf{a}
\subsection{Exercise 23}
\label{sec-2-23}
\begin{equation}
\begin{split}
& 1.\; (C \subset A) \implies (\exists B. ((B \subseteq A) \land (B \not \equiv A))) \\
& 2.\; \forall A. \exists C. ((A \in C) \land (C \not \equiv A))
\end{split}
\end{equation}

\emph{Answer:} \textbf{c}
\subsection{Exercise 24}
\label{sec-2-24}
\begin{equation}
\begin{split}
& 1.\; \Box (R(\mathbb{N}, \mathbb{N} \cup \{0\}) \land (\forall x \in \mathbb{N}. (x, x - 1) \in R)) \\
& 2.\; \mathbb{N} \cup \{\mathbb{N}\} \equiv P(\mathbb{N})
\end{split}
\end{equation}

\emph{Answer:} \textbf{d}
% Emacs 24.3.50.1 (Org mode 8.2.2)
\end{document}