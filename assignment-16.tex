% Created 2015-01-17 Sat 23:36
\documentclass[11pt]{article}
\usepackage[utf8]{inputenc}
\usepackage[T1]{fontenc}
\usepackage{fixltx2e}
\usepackage{graphicx}
\usepackage{longtable}
\usepackage{float}
\usepackage{wrapfig}
\usepackage{rotating}
\usepackage[normalem]{ulem}
\usepackage{amsmath}
\usepackage{textcomp}
\usepackage{marvosym}
\usepackage{wasysym}
\usepackage{amssymb}
\usepackage{hyperref}
\tolerance=1000
\usepackage[utf8]{inputenc}
\usepackage[usenames,dvipsnames]{color}
\usepackage[backend=bibtex, style=numeric]{biblatex}
\usepackage{commath}
\usepackage{tikz}
\usetikzlibrary{shapes,backgrounds}
\usepackage{marginnote}
\usepackage{listings}
\usepackage{color}
\usepackage{enumerate}
\hypersetup{urlcolor=blue}
\hypersetup{colorlinks,urlcolor=blue}
\addbibresource{bibliography.bib}
\setlength{\parskip}{16pt plus 2pt minus 2pt}
\definecolor{codebg}{rgb}{0.96,0.99,0.8}
\definecolor{codestr}{rgb}{0.46,0.09,0.2}
\author{Oleg Sivokon}
\date{\textit{<2015-01-17 Sat>}}
\title{Assignment 16, Introduction To Mathematics}
\hypersetup{
  pdfkeywords={Introduction To Mathematics, Assignment, Set Theory},
  pdfsubject={Sixth asssignment in the course Introduction To Mathematics},
  pdfcreator={Emacs 25.0.50.1 (Org mode 8.2.2)}}
\begin{document}

\maketitle
\tableofcontents


\lstset{ %
  backgroundcolor=\color{codebg},
  basicstyle=\ttfamily\scriptsize,
  breakatwhitespace=false,         % sets if automatic breaks should only happen at whitespace
  breaklines=false,
  captionpos=b,                    % sets the caption-position to bottom
  commentstyle=\color{mygreen},    % comment style
  framexleftmargin=10pt,
  xleftmargin=10pt,
  framerule=0pt,
  frame=tb,                        % adds a frame around the code
  keepspaces=true,                 % keeps spaces in text, useful for keeping indentation of code (possibly needs columns=flexible)
  keywordstyle=\color{blue},       % keyword style
  showspaces=false,                % show spaces everywhere adding particular underscores; it overrides 'showstringspaces'
  showstringspaces=false,          % underline spaces within strings only
  showtabs=false,                  % show tabs within strings adding particular underscores
  stringstyle=\color{codestr},     % string literal style
  tabsize=2,                       % sets default tabsize to 2 spaces
}

\clearpage

\section{Problems}
\label{sec-1}

\subsection{Problem 1}
\label{sec-1-1}
Given a system of axioms concerning ``points'', ``lines'' and a relation
``on'' \emph{(point on a line)}:

\begin{enumerate}
\item Given two distinct points $A$ and $B$, and two distinct lines $\ell_1$
and $\ell_2$ such that $A$ and $B$ are on both of them.
\item For every line $\ell$ there is a point $P$ which is not on $\ell$.
\begin{itemize}
\item Prove that the system is consistent.
\item Prove that the system is not categorical.
\item Prove that the system is independant.
\item Prove that the system entails theorem ``There exist at least four points''.
\end{itemize}
\end{enumerate}

\subsubsection{Answer 1}
\label{sec-1-1-1}
In order to show consistency it is enough to find a model for the system.
Let's find such a model.  If I consider lines as sets, then I can imagine
a system with lines $\ell_1=\{A,B,P_1\}$ and $\ell_2=\{A,B,P_2\}$.
By construction points $A,B$ are on both
$\ell_1$ and $\ell_2$, $P_1 \not \in \ell_2$ and $P_2 \not \in \ell_1$.
This satisfies both axioms, hence the model is consistent.
\subsubsection{Answer 2}
\label{sec-1-1-2}
In order to show that the system is not categorical I need to find at least
two models not isomorphic to each other.  Observe that we can easily extend
the model given in \ref{sec-1-1-1} by appending points to either of two lines
without violating any of the two axioms.  Thus, for example, a model
$\ell_1=\{A,B,P_1,P_3\},\ell_2=\{A,B,P_2\}$ would still satisfy both axioms
but would not be isomorphic to the system I extended because the first one
didn't have point $P_3$ in it.
\subsubsection{Answer 3}
\label{sec-1-1-3}
To show independence, I need to find such models where one of the axioms
in the system in question doesn't hold, but other do.  An example of such
system could be modeled using the model from the \ref{sec-1-1-1} with a line $\ell_3$
appended to it such that $\ell_3=\{A,B,P_1,P_2\}$.  This would allow for
the first axiom to hold, while the second axiom would be violated (as there
would be no points that are not on $\ell_3$.
\subsubsection{Answer 4}
\label{sec-1-1-4}
I am given at least two points by definition, now I need to establish that
there must be at least two more.  Observe that it is required by the second
axiom that there be a point $P$ not on some line.  Neither $A$ nor $B$ can
satisfy this condition, since both of them are on both lines, thus there
must be at least one more point for both $\ell_1$ and $\ell_2$ (since these
are the only lines warranted to exist by the first axiom).  Thus we have
that there will be at least four points in this system.
\subsection{Problem 2}
\label{sec-1-2}
Given the system of axioms which defins ``point'', ``line'', an ``on''
relation and axioms:
\begin{enumerate}
\item There are at least two lines.
\item There are exactly seven points.
\item There are exactly three points on each line.
\item For every two lines there's exactly one point that is on both of them.
\begin{itemize}
\item Prove that the system is consistent.
\item Prove that the system is independent.
\item Prove or disporve the statement: the system is categorical.
\end{itemize}
\end{enumerate}
Given following axioms:
\begin{enumerate}
\item Every two points are on one and only line.
\item Every three points are on one and only line.
\end{enumerate}

For axioms 5 and 6, prove or disprove: after appending either one of them
to the original system, the system will remain consistent.

\subsubsection{Answer 5}
\label{sec-1-2-1}
To prove consistency I will need to show a model for the system.  I will
start by naming the points: $A,B,C,D,E,F,G$.  Suppose now, I built four
three lines: $\ell_1=\{A,B,C\}$, $\ell_2=\{A,D,E\}$, $\ell_3=\{B,D,F\}$,
$\ell_4=\{C,D,G\}$.

First axiom holds since there are at least two lines (four actually).
Second axiom holds since there are exactly seven points.  Third axiom
holds since ther are exactly three points on each line.  Fourth axiom
holds since there is exactly one point where each two lines intersect.
Let's list them in a table:

\begin{center}
\begin{tabular}{l|llll}
 & $\ell_1$ & $\ell_2$ & $\ell_3$ & $\ell_4$\\
\hline
$\ell_1$ &  &  &  & \\
$\ell_2$ & $A$ &  &  & \\
$\ell_3$ & $B$ & $D$ &  & \\
$\ell_4$ & $C$ & $D$ & $D$ & \\
\end{tabular}
\end{center}

\emph{(note that we only need to fill half the table, since it's symmetrical)}
\subsubsection{Answer 6}
\label{sec-1-2-2}
To prove independence, I need to show a model that is inconsisten with
all modles, but is consistent with the system lacking either one of the
axioms.  First axiom warrants us the existence of lines, if it wasn't
for the first axiom, the system could have no lines, and all other
axioms would hold vacuously.  If we haven't been given that thre must
be exactly seven points, we could construct a system which would have
fewer points but satisfy all other conditions, for example:
$\ell_1=\{A,B,C\},\ell_2=\{C,D,E\}$, which has at least two lines,
exactly three points on each line and $C$ is the only line where $\ell_1$
and $\ell_2$ intersect.  If we remove the fourth axiom, then we could
build a model, where some lines don't intersect, for example:
$\ell_1=\{A,B,C\},\ell_2=\{C,D,E\},\ell_3=\{E,F,G\}$.

Since I've tried to remove every axiom and was able to build a model
inconsistent with the removed axiom, but consistent with the rest, the
system must be independent.
\subsubsection{Answer 7}
\label{sec-1-2-3}
The system is categorical.  It has exactly seven points and all
assignments of lines to these points are equivalent up to isomorphism.
To prove this, suppose this was not the case.  Suppose there was
an assignment of lines to points such that would be different from
the one modeled in \ref{sec-1-2-1}.  First, observe that this wouldn't have
been possible to achieve with three lines, as this would require that
provided there are three points on each line, either two of three lines
don't intersect, or, there would be one spare point left.  To see
why this is true, let's count the elements of three sets where each
contains exactly three elements such that it shares one of the elements
with the other set.  Since there are three of them, there must be two
elements in each set that are shared with the other sets.  I.e. each
set contains one element it doesn't share with anyone, and two elements
it shares with each other set.  This, in turn gives us 3 non-shared
elements, and three elements shared between pairs of sets.  Which is
one element shy of the requirement.

Observe also that this system cannot be modeled by five or more lines
because five lines would require that lines with exactly three points
intersect with every other four lines in exactly one point, which is
not possible (by pigeonhole principle, some two lines would have to
intersect in two points, or there would have to be more points in each
line).

With this out of the way, we can now limit our proof to only proving
that any assignment of four lines to seven points with initial constraints
will be isomorphic.  Obseve that without loss of generality, we could
select any three points and draw a line through them.  Let these be
$A$, $B$ and $C$ for the ease of reference.  We are now left with
four points, we need to draw a line through each pair of these four
points such that the third point would be then either $A$, $B$ or $C$.
Combinations of two from four give us three, all of them equivalent
choices.  These are also the only choices that we can make, since if
we chose either three or one points from the $D,E,F,G$, then we will
fail to satisfy either the requirement that all lines have a commont
point or that that any two lines intersect in only one point.
This gives us an assignment of lines to points unique up to renaming,
which is another way of saying that the model is unique up to isomorphism.
\subsubsection{Answer 8}
\label{sec-1-2-4}
Every two points are on one and only line---this follows from the system
in question.  Since we proved earlier that the system is categorical,
we can use its model to prove simply by counting that every line has
a combination of two unique points.
\subsubsection{Answer 9}
\label{sec-1-2-5}
If, however, we claimed that every three points are on one and only
line, the system would've become inconsistent.  One way to see that
is to, again, simply by looking at the model (and we are allowed to
do so, since we know it's the only possible model).  Another way
is to simply count the combinations of three out of seven without
repetition.  This gives us that if this axiom was consistent with
the system, there would have to be 35 lines, but we know ther to be
only 4.  Thus it would be a contradiction.
\subsection{Problem 3}
\label{sec-1-3}
Given the axioms of group which define ``element'' and ``binary operation''.
\begin{enumerate}
\item Prove that the system is consistent.
\item Prove that the second axiom is not derivable from the rest.
\item Prove that the fourth axiom is not derivable from the rest.
\item With added fifth axiom: ``there are exactly four elements'', 
let $G$ be a set of functions $f,g,h,k : \{1,2,3,4\}\to\{1,2,3,4\}$
defined as: $f$ is the identity function, $g(4)=4, g(3)=3, g(2)=1, g(1)=2$,
$h(4)=3, h(3)=4, h(2)=1, h(1)=2$, $k=g \circ h$.
Prove that $G$ with function composition operation models the system
$(1,2,3,4,5)$.
\item Prove that $(1,2,3,4,5)$ is not categorical.
\end{enumerate}

\subsubsection{Answer 10}
\label{sec-1-3-1}
To prove consistency I need to show a model of the system.  Let's
recal the axioms:
\begin{description}
\item[{Closure}] the result of application of group operation to any two
elements of the group is an element of the group.
\item[{Associativity}] the order of application of group operation doesn't
matter.
\item[{Identity element}] there exists an element in the group, which,
under group operation with any element of the group will give that
same element.
\item[{Inverse}] for every group element there exists an element in the
group which under group operation produces the identity element of
that group.
\end{description}

Integers under addition form a group, which is a model of any group,
thus the system is consistent.
\subsubsection{Answer 11}
\label{sec-1-3-2}
I don't know which one is second, but I'll assume \textbf{associativity}.
An example of a set with operation, which is closed, has identity element
and has inverses one could consider matrix multiplication on the set
of invertible matrices.  These will have inverses by construction,
multiplication of any two invertible matrices is still an invertible matrix,
the identity matrix is certainly the member of the set of all invertible
matrices, but, in general, multiplication will not be associative.
Thus, second axiom is not derivable from the rest.
\subsubsection{Answer 12}
\label{sec-1-3-3}
Again, I'll assume that the fourth axiom is the \textbf{inverse element}.
An example of an operation on a set which has closure, associativity,
identity element but not inverses is the set of natural numbers under
multiplication.  1 is the identity element, associativy holds by the
definition of multiplication and similarly for closure, however, in
general, natural numbers don't have inverses under multiplication in
the set of natural numbers.  (Only identity element has an inverse---itself).
\subsubsection{Answer 13}
\label{sec-1-3-4}
Let's prove $G$ is a group, since the fifth axiom holds trivially by
construction.

$G$ has identity element, again, by definition, it is $f$.  So,
the \textbf{identity} axiom holds. Now, observe that under function composition
$g \circ g = I$, $f \circ f = I$, $h \circ h = I$, $k \circ k = I$.
This warrants us that every element has an \textbf{inverse} (itself).  By looking
through all results of all functions, it is easy to see that none of the
given functions never produces anything other than 1, 2, 3 or 4, so $G$
is \textbf{closed} under function composition.

In order to prove \textbf{associativity} I will have to build the operation
table.  Given the table is symmetrical along its diagonal, the operation
would have to be associative:

\begin{center}
\begin{tabular}{l|llll}
 & $f$ & $g$ & $h$ & $k$\\
\hline
$f$ & $(1,2,3,4)$ & $(2,1,3,4)$ & $(1,2,4,3)$ & $(2,1,4,3)$\\
$g$ & $(2,1,3,4)$ & $(1,2,3,4)$ & $(2,1,4,3)$ & $(1,2,4,3)$\\
$h$ & $(2,1,3,4)$ & $(2,1,4,3)$ & $(1,2,3,4)$ & $(2,1,3,4)$\\
$k$ & $(2,1,4,3)$ & $(1,2,4,3)$ & $(2,1,3,4)$ & $(1,2,3,4)$\\
\end{tabular}
\end{center}

\emph{(Note that the results are given as tuples, where order matters)}

Since the table is symmetrical along its diagonal, the operation is associative.
This completes the proof.
\subsection{Problem 4}
\label{sec-1-4}
Given system of axioms which defines ``point'', ``line'' and ``on''
relationship where:
\begin{enumerate}
\item There are exactly four points.
\item Every two points belong to the one and only line.
\item For every line $\ell$ and every point $P$, which is not on $\ell$
there exists the only line such that $P$ is on it and this (other)
line has no points in common with $\ell$.

\item Prove that the system is consistent.
\item Prove that the system isn't categorical.
\item Prove that the system is not complete, in other words, that there
exists a theorem, which, when added to the system doesn't make it
contradictory.
\item Prove that this system entails ``there doesn't exist a line which
has exactly three points on it.''
\end{enumerate}

\subsubsection{Answer 14}
\label{sec-1-4-1}
To prove consistency, I will demonstrate the model for this system.
One such model can be constructed from
$\ell_1=\{A,B\},\ell_2=\{B,C\},\ell_3=\{C,D\},\ell_4=\{D,A\},\ell_5=\{A,C\}\ell_6=\{B,D\}$.

(1) holds because there are exactly four points: $A,B,C$ and $D$.
(2) holds because every two points belong to distinct lines.
(3) holds because for $\ell_1$ there's $\ell_3$ which satisfies
the condition for $C$ and $D$, which are not on $\ell_1$, and,
symmetrically, for all lines other than $\ell_5$ and $\ell_6$.
$\ell_6$ satisfies third axiom using $\ell_5$ and points $D$ and $B$,
$\ell_5$ is symmetrical to $\ell_6$.
\subsubsection{Answer 15}
\label{sec-1-4-2}
To prove that the system isn't categorical, I will build a model
which also satisfies the three axioms, but is not isomorophic to
the model given in \ref{sec-1-4-1}.  The model is very simple:
$\ell=\{A,B,C,D\}$.

(1) holds by construction.
(2) holds since any two points are on $\ell$ (there are no other
lines they can belong to).
(3) holds vacuously since there are no points which are not on $\ell$,
we are allowed to conclude that the condition is satisfied.
\subsubsection{Ansswer 16}
\label{sec-1-4-3}
There could be plenty of axioms added to this system such that they
will neither derive from the rest nor contradict the rest.  As
seen in \ref{sec-1-4-2} I could add a requirement that there be only
one line.  Since in the \ref{sec-1-4-1} I just demonstrated a model which
is consistent with the original model, but is inconsistent with
the new model, the added axiom is independentent of the first three.
The argument for consistency was given in \ref{sec-1-4-2}.
\subsubsection{Answer 17}
\label{sec-1-4-4}
In order to prove that the statement is not entailed by the system I
will add this statement to the system, and will prove inconsistency.

So, suppose there was line with exactly three points.  Without loss of
generality, let's name these points $A$, $B$ and $C$.  The remaining
point $D$ would, by the third axiom require that we be able to draw a line
through it, which has no common points with $\ell$ ($\ell=\{A,B,C\}$).
But we are also given that every two ponts must be on a line.  So, $D$
must be on a line with some other point.  Since we are given that there
are only four of them, we are left with just three choices, and all of
them are on $\ell$.  Thus $D$ must be on $\ell$.  But, by assumption,
$D$ is not on $\ell$.  This is a contradiction, hence system with
the fourth axiom added is inconsistent, hence it does not entail it.
% Emacs 25.0.50.1 (Org mode 8.2.2)
\end{document}