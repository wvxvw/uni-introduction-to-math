% Created 2014-11-22 Sat 22:01
\documentclass[11pt]{article}
\usepackage[utf8]{inputenc}
\usepackage[T1]{fontenc}
\usepackage{fixltx2e}
\usepackage{graphicx}
\usepackage{longtable}
\usepackage{float}
\usepackage{wrapfig}
\usepackage{rotating}
\usepackage[normalem]{ulem}
\usepackage{amsmath}
\usepackage{textcomp}
\usepackage{marvosym}
\usepackage{wasysym}
\usepackage{amssymb}
\usepackage{hyperref}
\tolerance=1000
\usepackage[utf8]{inputenc}
\usepackage[usenames,dvipsnames]{color}
\usepackage[backend=bibtex, style=numeric]{biblatex}
\usepackage{commath}
\usepackage{tikz}
\usetikzlibrary{shapes,backgrounds}
\usepackage{marginnote}
\usepackage{minted}
\usepackage{enumerate}
\usemintedstyle{perldoc}
\hypersetup{urlcolor=blue}
\hypersetup{colorlinks,urlcolor=blue}
\addbibresource{bibliography.bib}
\setlength{\parskip}{16pt plus 2pt minus 2pt}
\definecolor{codebg}{rgb}{0.96,0.99,0.8}
\author{Oleg Sivokon}
\date{\textit{<2014-11-22 Sat>}}
\title{Assignment 13, Introduction To Mathematics}
\hypersetup{
  pdfkeywords={Introduction To Mathematics, Assignment, Set Theory},
  pdfsubject={Second asssignment in the course Introduction To Mathematics},
  pdfcreator={Emacs 25.0.50.1 (Org mode 8.2.2)}}
\begin{document}

\maketitle
\tableofcontents


\clearpage

\section{Problems}
\label{sec-1}

\subsection{Problem 1}
\label{sec-1-1}

\begin{enumerate}
\item Given $G$ is a group under $\circ$ and $(\forall x, y \in G): x \circ y \circ x = y$.
Show that every member of $G$ is its own inverse and that $G$ is commutative.

\item Given $G$ is a group under $\circ$, $x, y \in G$ and $x$ being the inverse of 
      $x \circ y$, show that $G$ is commutative.
\end{enumerate}

\subsubsection{Answer 1}
\label{sec-1-1-1}

First, we will show that every element of $G$ is its own inverse.  By definition of
invertibility, $y \circ y^{-1} = 1$, 1 being the identity element.  Thus:

\begin{equation*}
  \begin{array}{lll}
    x \circ y \circ x = y & \iff &
    \textrm{given} \\
    x \circ (y \circ y^{-1}) \circ x = y \circ y^{-1} & \iff &
    \textrm{using group cancellation property} \\
    x \circ x = 1 & &
    \textrm{by definition of identity element} \\
  \end{array}
\end{equation*}

Since we selected $x$ arbitrary, it follows that
$(\forall x \in G): x \circ x = 1$, in other words, each element is its own inverse.

Having shown that $(\forall x \in G): x = x^{-1}$, we can use this fact to show that
$G$ is commutative.  Since $(x \circ y)$ is in $G$, it must be that
$(x \circ y) \circ (x \circ y)^{-1} = 1$ (1 being the identity element).  By
associativity, we can remove the parenthesis, and use the fact that each element
is its own inverse: $x \circ y \circ x \circ y = 1$, (*).

\begin{equation*}
  \begin{array}{llll}
    x \circ y & = & x \circ (1 \circ y)                           & \textrm{$1 = 1 \circ y$} \\
              & = & x \circ (x \circ y \circ x \circ y) \circ y   & \textrm{invoke (*)} \\
              & = & (x \circ x) \circ y \circ x \circ (y \circ y) & \textrm{by associativity} \\
              & = & y \circ x                                     & \textrm{completes the proof} \\
  \end{array}
\end{equation*}
\subsubsection{Answer 2}
\label{sec-1-1-2}

In order to prove commutativity w need to show that $y \circ x = x \circ y$.
We will start by restating the problem \emph{(1 is the identity element)}:

\begin{equation*}
  \begin{array}{lll}
    (x \circ y) \circ x = 1                  & \iff &
    \textrm{given} \\
    (x \circ y) \circ x = x \circ x^{-1}      & \iff &
    \textrm{by definition of identity} \\
    (x \circ y) \circ x = x \circ (x \circ y) & \iff &
    \textrm{from $x^{-1} = (x \circ y)$} \\
    x \circ (y \circ x) = x \circ (x \circ y) & \iff &
    \textrm{by associativity} \\
    y \circ x = x \circ y                     &      &
    \textrm{by group cancellation property} \\
  \end{array}
\end{equation*}

Having showed that $y \circ x = x \circ y$ we completed the proof.
\subsubsection{Answer 3}
\label{sec-1-1-3}
(ii) is flase, we can certainly have sets where intersection of that set with
the set of its subsets will produce no results.  In fact any set that contains
no other sets will produce an empty result under intersection.  For example:

Assume $C = \{1\}$, then:

\begin{equation}
\begin{split}
& P(C) = \{\emptyset, \{1\}\} \\
& P(C) \cap C = \emptyset
\end{split}
\end{equation}

We assumed $P(C) \cap C \neq \emptyset$, but discovered that it is also
possible that $P(C) \cap C = \emptyset$, which is a contradiction.
Therefore the initial claim is false.

\subsection{Problem 2}
\label{sec-1-2}
Given sets $A$, $B$, and $C$ prove that:

\begin{enumerate}
\item $(A \cup B) \setminus C = (A \setminus C) \cup (B \setminus C)$.
\item $A \setminus (B \setminus C) \subseteq (A \setminus B) \setminus C \implies A \cap C = \emptyset$.
\end{enumerate}

Prove or disprove:

\begin{enumerate}
\item $P(A) \subseteq B \implies \{\emptyset, A\} \in P(B)$.
\item $\{\emptyset, A\} \in P(B) \implies P(A) \subseteq B$.
\end{enumerate}

\subsubsection{Answer 4}
\label{sec-1-2-1}
I will prove this using a method similar to proving the distributivity property of
union over intersection.

To prove by contradiction, we need to examine two cases:

\begin{enumerate}
\item There is $x \in (A \cup B) \setminus C$ such that 
       $x \not \in ((A \setminus C) \cup (B \setminus C))$.
\item There is $x \in ((A \setminus C) \cup (B \setminus C))$ such that
       $x \not \in (A \cup B) \setminus C$.
\end{enumerate}

(1) For $x$ to be member of a difference of sets would mean it has to be the
member of minuend (by definition of subtraction).  We find that minuend is
the union of sets, which means that, by definition of union, $x$ could come
from either one of the unioned sets.  So, for $x$ not to be in the left-hand side
of the formula it would mean that it must be the case that it is either not in
the union of $A$ and $B$, or that it is in $C$.  Neither of which is true, which
is easy to become convinced of by observing the two parts of the right-hand side
of the formula: if $x \not \in (A \setminus C)$ then it absolutely has to be in
$B$ and not in $C$, which is exactly what the rest of the formula negates.
Thus we arrive at contradiction.  It must be the case that whenever
$x \in (A \cup B) \setminus C$ it is also the case that
$x \in ((A \setminus C) \cup (B \setminus C))$.

(2) By symmetrical argument we show that if
$x \in ((A \setminus C) \cup (B \setminus C))$ then it is the case that
$x \in (A \cup B) \setminus C$.

Having proved both cases we now conclude that union is distributive over
subtraction.
\subsubsection{Answer 5}
\label{sec-1-2-2}
To prove the claim, I'm going to show that there is no such $x$, for which
both:
$x \in (A \setminus (B \setminus C) \subseteq (A \setminus B) \setminus C)$
and $x \in (A \cap C)$ is true at the same time.  Proving this is equivalent
to proving the initial claim because if this was the case, then $A \cap C$
would not be an empty set (it would had to contain $x$), or
$A \setminus (B \setminus C) \subseteq (A \setminus B) \setminus C$ would be
false (these are the only two cases how a material implication can fail, from
definition of material implication).

For $x$ to be a member of $A \cap C$ means that it is both in $A$ and in $C$
(by definition of intersection).  This can only happen when we \emph{don't}
subtract $C$ from the result (otherwise the $A \cap C$ would be an empty set).
This only happens in $A \setminus (B \setminus C)$.  This means that if the
initial claim was true, there had to be such elements in
$A \setminus (B \setminus C)$, which are not in $(A \setminus B) \setminus C$,
but this contradicts the definition of set inclusion.  Since we arrived at
contradiction in our initial claim, which we set to disprove, we arrived at
the proof.
\subsubsection{Answer 6}
\label{sec-1-2-3}
Both $A$ and $\emptyset$ are certainly the members of $P(A)$, so if $B$
contains at least all the elements of $P(A)$, it certainly contains both the
$A$ and the $\emptyset$.  Then, of course, $P(B)$ must contain a set
containing two elements of $P(A)$, $\{A, \emptyset\}}$ among them.
\subsubsection{Answer 7}
\label{sec-1-2-4}
This is a false claim, here's an assignment that holds for antecedent, but
doesn't for consequent:

\begin{equation}
\begin{split}
& A = \{a, b\} \\
& P(A) = \{\emptyset, \{a\}, \{b\}, \{a, b\}\} \\
& B = \{\emptyset, \{a, b\}\} \\
& P(B) = \{\emptyset, \{\emptyset\}, \{\{a, b\}\}, \{\emptyset, \{a, b\}\}\} \\
& \{A, \emptyset\} \in P(B) = \{\emptyset, \{a, b\}\} \in P(B) =
\{\emptyset, \{a, b\}\} \in \{\emptyset, \{\emptyset\}, \{\{a, b\}\},
\{\emptyset, \{a, b\}\}\} \\
& P(A) \not \subseteq B = \{\emptyset, \{a\}, \{b\}, \{a, b\}\} \not \subseteq
\{\emptyset, \{a, b\}\}
\end{split}
\end{equation}

I.e. it is possible to construct such set $B$ that its powerset would satisfy
the left-hand side of the formula, but fail the right-hand side.

\subsection{Problem 3}
\label{sec-1-3}
\begin{enumerate}
\item Given a binary operation * on even numbers $A = \{2x | x \in \mathbb{N}\}$, for
all $x, y \in A$ the following holds:

\begin{equation}
x * y = \frac{(x - 2)(y - 2)}{2} + 2
\end{equation}

Verify whether $A$ is closed under *, that * is associative, that there exists
an identity element and that every element has its inverse under this operation.
\item Assume now that $A$ is defined as $A = \{2x | x \in \mathbb{Q}\}$, give the 
characteristics of properties outlined in (1).
\end{enumerate}

\subsubsection{Answer 8}
\label{sec-1-3-1}
In order to show that $A$ is closed under *, observe that adding 2 to any
even number will not make it odd.  We can thus disregard the addition.
Next, observe that in order for division by 2 to produce an even number the
divident must be divisible by 4.  Our dividen is a product of two numbers.
If we can show that both of them are even (that is each of them has a factor
of two), then their product must be a multiple of 4.  This problem is thus
equivalent to showing that even numbers are closed under subtraction of 2.
Given our definition of $A$, it is possible to produce 0 as the result of
the $(x - 2)(y - 2)/2$, if either $x$ or $y$ are equal to 2, however, the +2
part saves us from ever obtaining a final result less then 2.  Thus, I
conclude that $A$ is closed under *.

To prove $(x * y) * z = x * (y * z)$ observe that the problem scales down to
proving associativity of multiplication (since other function terms have no effect
on associativity property).  I.e. we actually need to prove
$((x - 2)(y - 2))(z - 2) = (x - 2)((y - 2)(z - 2))$ which is true, so long
multiplication on natural numbers is associative.

The definition of identity element is $ix = x$. And there is such element,
namely 4.  Since the function behaves similarly to multiplication, I looked
for the transformation I'd had to perform on the multiplication identity
element in order to find one here.  But, I wouldn't know of a proof of it
being the identity element.

Not all elements have their inverses in $A$. Inverse is defined to be $x * y = i$,
where $i$ is the identity element, whose existence we just established.  But
and since this function behaves equivalently to multiplication, we can assume
that its values only increase, therefore 6 * 2 = 2, 6 * 4 = 6, 6 * 6 = 10, \ldots{}
I.e. there would be no inverse for 6.  Another way to think about the problem
is to notice that 4, the identity element has no odd factors, but some even
numbers do have such factors.  By using only multiplication and no rationals less
then 1, we cannot possibly arrive from a number with odd factors to a number which
doesn't have them.
\subsubsection{Answer 9}
\label{sec-1-3-2}
The closure property will hold if we substitute $\mathbb{Q}$ for $\mathbb{N}$.
Even though we now allow negative numbers, we are no longer constrained by natural
numbers range.  Using this function we never create irrational numbers, because
all its input and its terms are integers.

Associativity holds just as it did before, we didn't rely on the properties of
integers to show the associativity the first time.

Consequently, the identity elements remains the same element.

However, now we have a chance of finding inverses for all elements!

It looks like the function:
\begin{equation}
x *' y = \frac{1}{2} - \frac{2}{(x + 2)(y + 2)}
\end{equation}
Will inverse the effect of the original definition, but it doesn't, becuase when
either $x = -2$ or $y = -2$ it results in division by zero.
\section{Exercises}
\label{sec-2}
Given $A = \{a\}$, and a binary operation defined on it such that $a * a = a$.
$B = \{a, b\}$ and binary operation + defined on $B$ as follows:

\begin{center}
\begin{tabular}{lll}
+ & a & b\\
\hline
a & a & b\\
b & b & b\\
\end{tabular}
\end{center}

$C = \{a, b, c\}$  and binary operation \% defined on $C$ as follows:

\begin{center}
\begin{tabular}{llll}
\% & a & b & c\\
\hline
a & a & b & c\\
b & b & a & c\\
c & c & a & a\\
\end{tabular}
\end{center}

$D$ is a non-empty set, $n, m \in \mathbb{N}$, $exp$ is the exponentiation function
defined on $\mathbb{N}$. $f$ defined as follows:

\begin{equation}
f(x, y) = xy \bmod 10
\end{equation}

$G$ is a group, with $\circ$ being its group operation, $e$ being its identity
element, having $a$, $b$ and $c$ as it members (it is also possible that $G$
has other members), $Commutative(S, O)$ predicate meaning that set $S$, its argument,
is commutative under operation $O$.

Answer the following questions using:

\begin{itemize}
\item \textbf{a} if only the first statement is correct.
\item \textbf{b} if only the second statement is correct.
\item \textbf{c} if both statements are correct.
\item \textbf{d} if neither statement is correct.
\end{itemize}

\subsection{Exercise 1}
\label{sec-2-1}
\begin{enumerate}
\item $A$ is closed under *.
\item * is not associative becuase $A$ has less then three members.
\end{enumerate}

\emph{Answer:} \textbf{a}
\subsection{Exercise 2}
\label{sec-2-2}
\begin{enumerate}
\item $a$ is the identity element of $A$ under *.
\item $A$ is a group under *.
\end{enumerate}

\emph{Answer:} \textbf{c}
\subsection{Exercise 3}
\label{sec-2-3}
\begin{enumerate}
\item $B$ is closed under +.
\item + is associative.
\end{enumerate}

\emph{Answer:} \textbf{c}
\subsection{Exercise 4}
\label{sec-2-4}
\begin{enumerate}
\item $b$ is the identity element in $B$.
\item $B$ is a group under +.
\end{enumerate}

\emph{Answer:} \textbf{d}
\subsection{Exercise 5}
\label{sec-2-5}
\begin{enumerate}
\item \% is associative.
\item There is identity element in $C$ under \%.
\end{enumerate}

\emph{Answer:} \textbf{b}
\subsection{Exercise 6}
\label{sec-2-6}
\begin{enumerate}
\item $b$ is the inverse of $c$ under \%.
\item $c$ is the inverse of $c$ under \%.
\end{enumerate}

\emph{Answer:} \textbf{b}
\subsection{Exercise 7}
\label{sec-2-7}
\begin{enumerate}
\item \% is commutative.
\item Every member of $A$ has an inverse under \%.
\end{enumerate}

\emph{Answer:} \textbf{c}
\subsection{Exercise 8}
\label{sec-2-8}
\begin{enumerate}
\item $P(D)$ is closed under $\cap$ (set intersection).
\item $\cap$ is associative in $P(D)$.
\end{enumerate}

\emph{Answer:} \textbf{c}
\subsection{Exercise 9}
\label{sec-2-9}
\begin{enumerate}
\item $\emptyset$ is the identity element of $P(D)$ under $\cap$.
\item $D$ has an inverse in $P(D)$ under $\cap$.
\end{enumerate}

\emph{Answer:} \textbf{b}
\subsection{Exercise 10}
\label{sec-2-10}
\begin{enumerate}
\item $P(D)$ is closed under $\setminus$ (set subtraction).
\item $\setminus$ is associative in $P(D)$.
\end{enumerate}

\emph{Answer:} \textbf{a}
\subsection{Exercise 11}
\label{sec-2-11}
\begin{enumerate}
\item $\emptyset$ is the identity element of $P(D)$ under $\setminus$.
\item $\emptyset$ is the identity element of $P(D)$ under $\cup$.
\end{enumerate}

\emph{Answer:} \textbf{c}
\subsection{Exercise 12}
\label{sec-2-12}
\begin{enumerate}
\item $exp$ is associative.
\item $exp$ has an identity element.
\end{enumerate}

\emph{Answer:} \textbf{b}
\subsection{Exercise 13}
\label{sec-2-13}
\begin{enumerate}
\item $\{2, 4, 6, 8\}$ is a group under $f$.
\item $\{1, 3, 5, 7, 9\}$ is a group under $f$.
\end{enumerate}

\emph{Answer:} \textbf{a}
\subsection{Exercise 14}
\label{sec-2-14}
\begin{enumerate}
\item $(x \in G \land x \circ x = x) \implies (x = e)$.
\item $(x \in G \land x \circ x = e) \implies (x = e)$.
\end{enumerate}

\emph{Answer:} \textbf{b}
\subsection{Exercise 15}
\label{sec-2-15}
\begin{enumerate}
\item $(a \circ b = b \circ a) \implies Commutative(G, \circ)$.
\item $(a = b^{-1}) \implies (b = a^{-1})$.
\end{enumerate}

\emph{Answer:} \textbf{d}
\subsection{Exercise 16}
\label{sec-2-16}
\begin{enumerate}
\item $\forall x, y, z \in G ((x \circ y = x \circ z) \implies (y = z))$.
\item $\forall x, y, z \in G ((x \circ y = y \circ z) \implies (x = z))$.
\end{enumerate}

\emph{Answer:} \textbf{d}
\subsection{Exercise 17}
\label{sec-2-17}
\begin{enumerate}
\item The diagonal of the table describing $\circ$ on $G$ all members of $G$ 
must appear.
\item $a$ appears exactly once in every column and every row of the table 
dscribing $\circ$ on $G$.
\end{enumerate}

\emph{Answer:} \textbf{d}
\subsection{Exercise 18}
\label{sec-2-18}
\begin{enumerate}
\item $(a \circ b \circ c)^{-1} = a^{-1} \circ b^{-1} \circ c^{-1}$.
\item $((a \circ b)^{-1} \neq a^{-1} \circ b^{-1}) \implies \lnot Commutative(G, \circ)$.
\end{enumerate}

\emph{Answer:} \textbf{a}
\subsection{Exercise 19}
\label{sec-2-19}
\begin{enumerate}
\item $(a \circ b = e) \implies (b \circ a = e)$.
\item If $G$ has exactly four members and $a \circ b = c$, then $b \circ a = c$.
\end{enumerate}

\emph{Answer:} \textbf{d}
\subsection{Exercise 20}
\label{sec-2-20}
\begin{enumerate}
\item Any group with one, two or three members is commutative.
\item Any group is commutative.
\end{enumerate}

\emph{Answer:} \textbf{a}
\subsection{Exercise 21}
\label{sec-2-21}
Let $A$ be a group with three members.
\begin{enumerate}
\item There should exist a binary operation that conforms to all requirements of 
group operation sans associativity.
\item There exists a binary operation on $A$ which doesn't support associativity
but has the cancellation property.
\end{enumerate}

\emph{Answer:} \textbf{b}
\subsection{Exercise 22}
\label{sec-2-22}
Let $A$ be a set with defined binary operation *, which conforms to all
group properties sans inverse element.
\begin{enumerate}
\item If * is commutative, then $A$ is a group under *.
\item If * supports the cancellation property, then $A$ is a group under *.
\end{enumerate}

\emph{Answer:} \textbf{d}
\subsection{Exercise 23}
\label{sec-2-23}
We define a binary operation on $\mathbb{N} \cup \{{1\over 2}\}$ thus:
$(\forall x, y \in \mathbb{N} \cup \{{1\over 2}\}): x \Delta y = 2xy$.
\begin{enumerate}
\item $\Delta$ has the cancellation property.
\item $\mathbb{N} \cup \{{1\over 2}\}$ is a group under $\Delta$.
\end{enumerate}

\emph{Answer:} \textbf{c}
\subsection{Exercise 24}
\label{sec-2-24}
Let $G$ be something defined elsewhere.
\begin{enumerate}
\item $(\exists x, y, z \in G): x \circ y = y \circ z$, but $x \neq z$.
\item $(\forall x \in G): x \circ x = I \lor x \circ x \circ x = I$.
\end{enumerate}

\emph{Answer:} \textbf{not a foggiest idea}
% Emacs 25.0.50.1 (Org mode 8.2.2)
\end{document}