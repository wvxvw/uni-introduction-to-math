% Created 2014-11-30 Sun 20:47
\documentclass[11pt]{article}
\usepackage[utf8]{inputenc}
\usepackage[T1]{fontenc}
\usepackage{fixltx2e}
\usepackage{graphicx}
\usepackage{longtable}
\usepackage{float}
\usepackage{wrapfig}
\usepackage{rotating}
\usepackage[normalem]{ulem}
\usepackage{amsmath}
\usepackage{textcomp}
\usepackage{marvosym}
\usepackage{wasysym}
\usepackage{amssymb}
\usepackage{hyperref}
\tolerance=1000
\usepackage[utf8]{inputenc}
\usepackage[usenames,dvipsnames]{color}
\usepackage[backend=bibtex, style=numeric]{biblatex}
\usepackage{commath}
\usepackage{tikz}
\usetikzlibrary{shapes,backgrounds}
\usepackage{marginnote}
\usepackage{listings}
\usepackage{color}
\usepackage{enumerate}
\hypersetup{urlcolor=blue}
\hypersetup{colorlinks,urlcolor=blue}
\addbibresource{bibliography.bib}
\setlength{\parskip}{16pt plus 2pt minus 2pt}
\definecolor{codebg}{rgb}{0.96,0.99,0.8}
\definecolor{codestr}{rgb}{0.46,0.09,0.2}
\author{Oleg Sivokon}
\date{\textit{<2014-11-22 Sat>}}
\title{Assignment 13, Introduction To Mathematics}
\hypersetup{
  pdfkeywords={Introduction To Mathematics, Assignment, Set Theory},
  pdfsubject={Second asssignment in the course Introduction To Mathematics},
  pdfcreator={Emacs 25.0.50.1 (Org mode 8.2.2)}}
\begin{document}

\maketitle
\tableofcontents


\lstset{ %
  backgroundcolor=\color{codebg},
  basicstyle=\ttfamily\scriptsize,
  breakatwhitespace=false,         % sets if automatic breaks should only happen at whitespace
  breaklines=false,
  captionpos=b,                    % sets the caption-position to bottom
  commentstyle=\color{mygreen},    % comment style
  framexleftmargin=10pt,
  xleftmargin=10pt,
  framerule=0pt,
  frame=tb,                        % adds a frame around the code
  keepspaces=true,                 % keeps spaces in text, useful for keeping indentation of code (possibly needs columns=flexible)
  keywordstyle=\color{blue},       % keyword style
  showspaces=false,                % show spaces everywhere adding particular underscores; it overrides 'showstringspaces'
  showstringspaces=false,          % underline spaces within strings only
  showtabs=false,                  % show tabs within strings adding particular underscores
  stringstyle=\color{codestr},     % string literal style
  tabsize=2,                       % sets default tabsize to 2 spaces
}

\clearpage

\section{Problems}
\label{sec-1}

\subsection{Problem 1}
\label{sec-1-1}

\begin{enumerate}
\item Given $G$ is a group under $\circ$ and $(\forall x, y \in G): x \circ y \circ x = y$.
Show that every member of $G$ is its own inverse and that $G$ is commutative.

\item Given $G$ is a group under $\circ$, $x, y \in G$ and $x$ being the inverse of 
      $x \circ y$, show that $G$ is commutative.
\end{enumerate}

\subsubsection{Answer 1}
\label{sec-1-1-1}

First, we will show that every element of $G$ is its own inverse.  By definition of
invertibility, $y \circ y^{-1} = 1$, 1 being the identity element.  Thus:

\begin{equation*}
  \begin{array}{lll}
    x \circ y \circ x = y & \iff &
    \textrm{given} \\
    x \circ (y \circ y^{-1}) \circ x = y \circ y^{-1} & \iff &
    \textrm{using group cancellation property} \\
    x \circ x = 1 & &
    \textrm{by definition of identity element} \\
  \end{array}
\end{equation*}

Since we selected $x$ arbitrary, it follows that
$(\forall x \in G): x \circ x = 1$, in other words, each element is its own inverse.

Having shown that $(\forall x \in G): x = x^{-1}$, we can use this fact to show that
$G$ is commutative.  Since $(x \circ y)$ is in $G$, it must be that
$(x \circ y) \circ (x \circ y)^{-1} = 1$ (1 being the identity element).  By
associativity, we can remove the parenthesis, and use the fact that each element
is its own inverse: $x \circ y \circ x \circ y = 1$, (*).

\begin{equation*}
  \begin{array}{llll}
    x \circ y & = & x \circ (1 \circ y)                           & \textrm{$1 = 1 \circ y$} \\
              & = & x \circ (x \circ y \circ x \circ y) \circ y   & \textrm{invoke (*)} \\
              & = & (x \circ x) \circ y \circ x \circ (y \circ y) & \textrm{by associativity} \\
              & = & y \circ x                                     & \textrm{completes the proof} \\
  \end{array}
\end{equation*}
\subsubsection{Answer 2}
\label{sec-1-1-2}

In order to prove commutativity we need to show that $y \circ x = x \circ y$.
We will start by restating the problem \emph{(1 is the identity element)}:

\begin{equation*}
  \begin{array}{lll}
    (x \circ y) \circ x = 1                  & \iff &
    \textrm{given} \\
    (x \circ y) \circ x = x \circ x^{-1}      & \iff &
    \textrm{by definition of identity} \\
    (x \circ y) \circ x = x \circ (x \circ y) & \iff &
    \textrm{from $x^{-1} = (x \circ y)$} \\
    x \circ (y \circ x) = x \circ (x \circ y) & \iff &
    \textrm{by associativity} \\
    y \circ x = x \circ y                     &      &
    \textrm{by group cancellation property} \\
  \end{array}
\end{equation*}

Having showed that $y \circ x = x \circ y$ we completed the proof.

\subsection{Problem 2}
\label{sec-1-2}

Let $H = \{a, b, c, e\}$, all four elements of $H$ being distinct, and *
being the binary operation on $H$.  $e$ is the identity element under *,
and $a*a=e$, $b*b=e$.

\begin{enumerate}
\item Prove that if $H$ has cancellation property, than it must be that $c*a \neq e$.
\item Prove that if * is associative, then $c*b \neq e$.
\item Prove that if $H$ is a group under *, then $c*c=e$.
\item Complete the operation table of $(H, *)$, assuming $H$ is a group.
\end{enumerate}

\subsubsection{Answer 3}
\label{sec-1-2-1}
It is possible to have at most one identity element in a monoid.  If there was
a $c$ such that both $c*a=e$ and $a*a=e$, the cancellation property would require
that $c=a$, but this is not possible, because we are given that $c \neq a$.
Hence $c*a \neq e$.
\subsubsection{Answer 4}
\label{sec-1-2-2}
We will again use the fact that all elements of $H$ must be distinct, and show
that $c*b \neq e$ if $H$ is associative.

\begin{equation*}
  \begin{array}{lll}
    c*b = e & \iff &
    \textrm{Assume the opposite} \\
    c*b=b*b & \iff &
    \textrm{Because $b*b=e$} \\
    (c*b)*b=(b*b)*b & \iff &
    \textrm{Apply *once again} \\
    c*(b*b)=b*(b*b) & \iff &
    \textrm{By associativity} \\
    c=b &  &
    \textrm{Contradiction, $c \neq b$} \\
  \end{array}
\end{equation*}

We assumed $c*b=e$, but derived $c=b$, which is a contradiction, thus
it must be the case that $c*b \neq e$. This completes the proof.
\subsubsection{Answer 5}
\label{sec-1-2-3}
In \ref{sec-1-2-1} and \ref{sec-1-2-2} we have shown that both $c*a \neq e$ and $c*b \neq e$.
Since by definition of the group every element must have an inverse, we are only
left with $c$ as a candidate.  $e$ is not a candidate because it is the identity
element, which means $c*e=c$.  So it must be the case that $c*c=e$.
\subsubsection{Answer 6}
\label{sec-1-2-4}
This table was constructed so that any pair of elements under * will produce
the third element (excluding the identity element).  In this way it is
ensured that $x * y * z = e$ for any elements of $H$, in any order.

\begin{center}
\begin{tabular}{l|llll}
 & a & b & c & e\\
\hline
a & e & c & b & a\\
b & c & e & a & b\\
c & b & a & e & c\\
e & a & b & c & e\\
\end{tabular}
\end{center}
\subsection{Problem 3}
\label{sec-1-3}
\begin{enumerate}
\item Prove that if in the group $G$ every element is its own inverse, then this
group is commutative.
\item Prove that the group $G = (\{0, 1, 2, 3, 4\}, \circ)$ where 
$x \circ y = mod(x + y, 5)$ is commutative, but no member of this group is
its own inverse, except for the identity element.
\item Give an example of a non-commutative group, such that it has an element 
which is its own inverse, and isn't its identity element.
\end{enumerate}

\subsubsection{Answer 7}
\label{sec-1-3-1}
The proof is essentially the same as \ref{sec-1-1-2}.  Let's first express the
identity element.  Let $x$ and $y$ be two arbitrary chosen members of $G$, then
it follows that $x*x=y*y$, $*$ being the group operation.  Consequently,
$x*y*y*x=1$ (1 being the identity element) because $x*x=1$, $y*y=1$ and $x*1*x=1$.

Now we will use this presentation of identity element to show that $x*y=y*x$:

\begin{equation*}
  x*y=x*1*y=x*x*y*y*x*y=1*1*x*y=x*y
\end{equation*}

This completes the proof.
\subsubsection{Answer 8}
\label{sec-1-3-2}
Commutativity of $G$ trivially follows from commutativity of addition.  It doesn't
matter whether we add $x+y$ or $y+x$ and then take the modulo of the sum, the sum
is guaranteed to be the same in both cases and modulo is left-concealable.

Let's establish the identity element of this group.  It must be 0 since $\mod 5$
slices the set of integers into five slices, each member of this group stands for
a distinct slice.  The remainder of the $x+0$, $x$ bein any element of the group
is thus $x$ itself.  Since we know that $G$ is commutative,
$x \circ 0 = 0 \circ x = x$.

Now we can use direct calculation to verify that no element of $G$ under $\circ$
is its own inverse (except for the identity element).

\begin{equation*}
  \begin{array}{lll}
    0 \circ 0 & = & 0 \\
    1 \circ 1 & = & 2 \\
    2 \circ 2 & = & 4 \\
    3 \circ 3 & = & 1 \\
    4 \circ 4 & = & 3 \\
  \end{array}
\end{equation*}
\subsubsection{Answer 9}
\label{sec-1-3-3}

The example of a group which is non-commutative, but has elements which are
their own inverses could be a dyhedral group of rank 6.  I found a mention
of this group on the internet, so I didn't come up with it myself.  I've
verified by direct calculation that it indeed meets the requirements.

\begin{center}
\begin{tabular}{l|llllll}
 & 1 & a & b & c & d & e\\
\hline
1 & 1 & a & b & c & d & e\\
a & a & 1 & c & b & e & d\\
b & b & d & 1 & e & a & c\\
c & c & e & a & d & c & b\\
d & d & b & e & 1 & c & a\\
e & e & c & d & a & b & 1\\
\end{tabular}
\end{center}

The calculation follows.

First define some utility functions:

\clearpage

\lstset{language=Lisp,numbers=none}
\begin{lstlisting}
(defun op (table elements)
  "Creates an operation from the TABLE describing the results
of this operation on the group of ELEMENTS."
  (lambda (a b)
    (loop :with result := (aref table (car a) (car b))
       :for (head . tail) :in elements :do
       (when (eql tail result)
         (return (cons head tail))))))

(defun elements (table)
  "Collects all the elements of the group whose operation is
defined in TABLE."
  (loop :for i :below (array-dimension table 0)
     :collect (cons i (aref table i 0))))
\end{lstlisting}

Next define the predictates validating different aspects of the
group.

\lstset{language=Lisp,numbers=none}
\begin{lstlisting}
(defun verify-associativity (operations-table)
  (loop
     :with elements := (elements operations-table)
     :with op := (op operations-table elements)
     :for x :in elements :do
     (loop :for y :in elements :do
        (loop :for z :in elements :do
           (unless (equal (funcall op (funcall op x y) z)
                          (funcall op x (funcall op y z)))
             (return-from verify-associativity
               (list x y z)))))))

(defun find-identity (operations-table)
  "Searches for identity element of the group defined by
OPERATIONS-TABLE."
  (loop
     :with elements := (elements operations-table)
     :with op := (op operations-table elements)
     :for x :in elements :do
     (loop :for y :in elements :do
        (unless (equal (funcall op x y) (funcall op y x))
          (return))
        :finally (return-from find-identity (cdr x)))))

(defun find-inverses (operations-table identity-element)
  "Searches for inverses of each element of the group given
by the OPERATIONS-TABLE.  This relies on the identity element
being previously calculated."
  (loop
     :with elements := (elements operations-table)
     :with op := (op operations-table elements)
     :for x :in elements :nconc
     (loop :for y :in elements
        :when (eql (cdr (funcall op x y)) identity-element)
        :collect (list (cdr x) (cdr y)))))
\end{lstlisting}

Similarly, define a predicate for commutativity:

\lstset{language=Lisp,numbers=none}
\begin{lstlisting}
(defun verify-commutativity (operations-table)
  "Verifies whether the group given by OPERATIONS-TABLE
 is commutative."
  (loop
     :with elements := (elements operations-table)
     :with op := (op operations-table elements)
     :for x :in elements :do
     (loop :for y :in elements :do
        (unless (equal (funcall op x y) (funcall op y x))
          (return-from verify-commutativity
            (list (cdr x) (cdr y) 
                  (cdr (funcall op x y))
                  (cdr (funcall op y x))))))))
\end{lstlisting}

Finally, print out the results:

\lstset{language=Lisp,numbers=none}
\begin{lstlisting}
(defun print-report ()
  "Prints the report fo dyhedral group of rank 6."
  (let* ((dyhedral-group-6
          (make-array 
           (list 6 6)
           :initial-contents
           '((1 a b c d e)
             (a 1 c b e d)
             (b d 1 e a c)
             (c e a d 1 b)
             (d b e 1 c a)
             (e c d a b 1))))
         (associativity (verify-associativity dyhedral-group-6))
         (identity (find-identity dyhedral-group-6))
         (inverses (find-inverses dyhedral-group-6 identity))
         (commutativity (verify-commutativity dyhedral-group-6)))
    (list
     (list "associativity" (null associativity))
     (list "identity" identity)
     (list "inverses" (format nil "~($~{~{~s*~s^{-1}~}=~}1$~)" 
                              inverses))
     (list "commutativity"
      (apply 'format nil "~($~s*~s=~s, ~3:*~s*~s=~*~s$~)" 
             commutativity)))))

(print-report)
\end{lstlisting}

\begin{center}
\begin{tabular}{ll}
associativity & T\\
identity & 1\\
inverses & $1*1^{-1}=a*a^{-1}=b*b^{-1}=c*d^{-1}=d*c^{-1}=e*e^{-1}=1$\\
commutativity & $a*b=c, a*b=d$\\
\end{tabular}
\end{center}
\subsection{Problem 4}
\label{sec-1-4}
Let $A = \{e, a, b, c, ...\}$, $e, a, b, c$ being distinct.  $*$ is a binary
operation defined on $A$, under which $A$ is closed, has cancellation property,
is associative.  $e$ is the identity element in $A$ under $*$ and $a$ is its
own inverse.

\begin{enumerate}
\item Prove that $B = \{e, a, b, a*b\}$ has four distinct members.
\item Prove that if $c \not \in B$, then $a*c \not \in B$.
\item Prove that in a group of five elements no element but the identity element
is its own inverse.
\end{enumerate}

\subsubsection{Answer 10}
\label{sec-1-4-1}
We are given that $e, a, b$ are distinct by construction, what is left to show
is that $a*b$ is equal to neither one of them.  Recall that we are given that
$*$ has cancellation property, this in other words means that $a*b \neq e$,
otherwise the cancellation property wouldn't hold (since it would be both
$a*a=e$ and $a*b=e$).

Suppose then, by contradiction that $a*b=a$, remembering that $*$ is
associtative we would have that $(a*a)*b=a*(a*b)$, but this is not the case
because $(a*a)*b=b$ and $a*(a*b)=e$, while we are given that $b \neq e$. So,
$a*b \neq a$.

Suppose, again, by contradiction that $a*b=b$, this would also imply that
$(a*a)*b=b$ and $(a*e)*b=b$, but we are given that $*$ has cancellation property,
which must mean that $a*a=a*e$, but we also know that $a*a=e$ and $a*e=a$,
thust it would have to be that $a=e$, but we are given that $a \neq d$.  This
means that $a*b \neq b$.

Sine we tried $a$, $e$ and $b$ and convinced that neither of them is a
candidate for $a*b$, we conclude that $a*b$ must be distinct fourth member of
$B$.  This completes the proof.
\subsubsection{Answer 11}
\label{sec-1-4-2}
This question becomes trivial, if you consider its negation:

\begin{equation*}
  \exists c \in B: a*c \not \in B
\end{equation*}

Disproving this would prove the original claim.  Observe now that $a*e$ is
in $B$ because $e$ is the identity element and $a \in B$, $a*b$ is in $B$ by
construction and $a*a$ is again in $B$ because we are given that $a$ is its
own inverse, meaning $a*a=e$ and $e \in B$.  Lastly, $a*(a*b)$ is in $B$
because by associativity, $a*(a*b)=(a*a)*b$, then $a$ being its own inverse
gives $(a*a)*b=b$, and $b$ is in $B$ by construction.

We exhausted all possibilities for $c$ and none satisfies the
existence condition, hence $\lnot (\exists c \in B: a*c \not \in B)$, hence
$\forall c \not \in B: a*c \not \in B$.
\subsubsection{Answer 12}
\label{sec-1-4-3}
Not exactly relevant to the question, I found that the number of groups of
different orders are extensively studied and are well-known for various
kinds of groups.  For instance, just by looking at \url{http://oeis.org/A000001}
I would know that there is only one group of order 5, and so by constructing
it, I'd have proved that there aren't elements in this group, which are
their own inverses.  But I would imagine this answer to not be satisfactory.

So, in order to prove this claim more generally, we can reuse the previous
answers.  So far we have worked with the group of size four and we were
given that it has one element which is its own inverse. It is easy to see
that $a*b$ must be in the group $G$ (I will base it on the earlier
definition of $B$).  And $c$ being either one of $b*a$, $a*b*b$, $b*b$ by
pigeonhole principle.

I will now to construct a Cayley table and show its inconsistency:

\[\begin{array}{l|lllll}
  & 1 & a & b & a*b & c \\
 1 & 1 & a & b & a*b & c \\
 a & a & 1 & a*b & b & a*c \\
 b & b & b*a & b*b & b*a*b & b*c \\
 a*b & a*b & a*b*a & a*b*b & a*b*a*b & a*b*c \\
 c & c & c*a & c*b & c*a*b & c*c \\
\end{array}\]

Look at the second row of this table, its last element $a*c$.  Observe that
while it has to be one of $\{1, a, b, a*b, c\}$ it is neither $1$, nor $a$ nor
$b$ nor $a*b$ (cancellation implies that all columns of the same row of the
Cayley table must have distinct values).  Then, it must be the case that
$a*c=c$, but the same cancellation property prevents us from having same
values in a single colum, but $a*c=c$ implies that $c$ would appear in its
column twice, which is a contradiction.

We selected $a$ arbitrarily to be its own inverse, since we arrived at
contradiction, we are free to assume that if there exists a group of rank
five, it should not have a member which is its own inverse.  Unfortunately
we did not construct such a group, so we aren't free to claim (yet) this
doesn't hold vacuously, but this should be enough to answer the question.
% Emacs 25.0.50.1 (Org mode 8.2.2)
\end{document}