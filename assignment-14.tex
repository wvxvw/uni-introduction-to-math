% Created 2014-12-20 Sat 19:02
\documentclass[11pt]{article}
\usepackage[utf8]{inputenc}
\usepackage[T1]{fontenc}
\usepackage{fixltx2e}
\usepackage{graphicx}
\usepackage{longtable}
\usepackage{float}
\usepackage{wrapfig}
\usepackage{rotating}
\usepackage[normalem]{ulem}
\usepackage{amsmath}
\usepackage{textcomp}
\usepackage{marvosym}
\usepackage{wasysym}
\usepackage{amssymb}
\usepackage{hyperref}
\tolerance=1000
\usepackage[utf8]{inputenc}
\usepackage[usenames,dvipsnames]{color}
\usepackage[backend=bibtex, style=numeric]{biblatex}
\usepackage{commath}
\usepackage{tikz}
\usetikzlibrary{shapes,backgrounds}
\usepackage{marginnote}
\usepackage{listings}
\usepackage{color}
\usepackage{enumerate}
\hypersetup{urlcolor=blue}
\hypersetup{colorlinks,urlcolor=blue}
\addbibresource{bibliography.bib}
\setlength{\parskip}{16pt plus 2pt minus 2pt}
\definecolor{codebg}{rgb}{0.96,0.99,0.8}
\definecolor{codestr}{rgb}{0.46,0.09,0.2}
\author{Oleg Sivokon}
\date{\textit{<2014-12-19 Fri>}}
\title{Assignment 14, Introduction To Mathematics}
\hypersetup{
  pdfkeywords={Introduction To Mathematics, Assignment, Set Theory},
  pdfsubject={Second asssignment in the course Introduction To Mathematics},
  pdfcreator={Emacs 25.0.50.1 (Org mode 8.2.2)}}
\begin{document}

\maketitle
\tableofcontents


\lstset{ %
  backgroundcolor=\color{codebg},
  basicstyle=\ttfamily\scriptsize,
  breakatwhitespace=false,         % sets if automatic breaks should only happen at whitespace
  breaklines=false,
  captionpos=b,                    % sets the caption-position to bottom
  commentstyle=\color{mygreen},    % comment style
  framexleftmargin=10pt,
  xleftmargin=10pt,
  framerule=0pt,
  frame=tb,                        % adds a frame around the code
  keepspaces=true,                 % keeps spaces in text, useful for keeping indentation of code (possibly needs columns=flexible)
  keywordstyle=\color{blue},       % keyword style
  showspaces=false,                % show spaces everywhere adding particular underscores; it overrides 'showstringspaces'
  showstringspaces=false,          % underline spaces within strings only
  showtabs=false,                  % show tabs within strings adding particular underscores
  stringstyle=\color{codestr},     % string literal style
  tabsize=2,                       % sets default tabsize to 2 spaces
}

\clearpage

\section{Problems}
\label{sec-1}

\subsection{Problem 1}
\label{sec-1-1}

Given $A = \{1, 2\}$ and $B = \mathbb{N}$.
\begin{enumerate}
\item Describe all functions from $A$ to $B$, which are not one-to-one (bijective).
\item Describe all functions from $B$ to $A$ which are not onto.
\item Prove or disprove:
\begin{itemize}
\item There exist functions $f : A \to B$ and $g : B \to A$ such that $f \circ g$
has an inverse.
\item There exist functions $f : A \to B$ and $g : B \to A$ such that $g \circ f$
has an inverse.
\end{itemize}
\end{enumerate}

\subsubsection{Answer 1}
\label{sec-1-1-1}

Let's first give a name for any function $f : A \to B$.  Now, any $f$, which sends
either of the elements of $A$ (of which there are two) to the same element of
$B$ would be non-bijective.  We can describe the set of all such functions using
this formula:

\begin{equation*}
  F = \{(a,b) |
  a \in A,
  b \in B,
  \exists a' \in A: a' \neq a \land f(a) = f(a') \}
\end{equation*}

To make this more concrete: all functions $f$ which assign the same value in $B$ to
either 1 or 2 are non-bijections.  This is assuming $f$ is defined for all inputs.
\subsubsection{Answer 2}
\label{sec-1-1-2}

Let's, again, give a name for all functions from $B$ to $A$, viz. $g : B \to A$.
The function is onto (surjective) iff every value in its codomain has an origin
in its domain.  We can describe all such functions using this formula:

\begin{equation*}
  G = \{(b,a) |
  a \in A,
  b \in B,
  \forall b' \in B: g(b) = g(b') \}
\end{equation*}

Or, in other words, all functions which send all elements of $B$ to the same
element of $A$ (be it 1 or 2), are not surjective.
\subsubsection{Answer 3}
\label{sec-1-1-3}
Yes, there exist such functions.  For example, $g$ may be defined to return
1 if the input is odd and 2, when the input is even.  Then, if $f$ sends
its input to an odd number, given 1 and to an even number, given 2, the
composition $f \circ g$ is simply the identity function.
\subsubsection{Answer 4}
\label{sec-1-1-4}
We, however, cannot define a $g \circ f$ function with an inverse.  The
intuition behind this is that we lose information by sending elments from a
larger domain to a smaller one.  More formally, we can procede by pigeonhole
principle, seeing how any function $g$ (provided it is defined for all
inputs) would have to send its input to the same element in $A$, while
afterwards there would be no way of distinguishing that element from the one
obtained by earlier applicatin of $g$.
\subsection{Problem 2}
\label{sec-1-2}
Given $f : A \to B$ and $C \subseteq A$.

\begin{enumerate}
\item Prove that $C \subseteq f^{-1}(f(C))$.
\item Prove that if $f$ is a bijection, then $C = f^{-1}(f(C))$.
\item Show sets $A$, $B$, $C$ and function $f:A \to B$ such that it holds that
      $C \subset f^{-1}(f(C))$.
\end{enumerate}

\subsubsection{Answer 5}
\label{sec-1-2-1}
Let's assume by contradiction, that there exists $d$ in $D = A \setminus C$,
such that it is in the codomain of $f^{-1}$.  Existence of sucn an element
would in turn imply existence of element $b \in B$ such that $f^{-1}(b) = d$.
This would in turn imply existence of $c \in C$ such that $f(c) = b$.
But by the definition of inverse, $f^{-1}(b)=c$ and $c \in C$, while we
assumed the contrary.  Since our initial assumption failed, it must be the
case that the elements of codomain of $f^{-1}$ must come from $C$.
\subsubsection{Answer 6}
\label{sec-1-2-2}
Since $f$ is a bijection, it's inverse must be a bijection too (otherwise it
would not be able to assign to all the elements in its image all the values in
its domain).  Composition of two bijections is necessarily a bijection, thus
$C = f^{-1}(f(C))$.
\subsubsection{Answer 7}
\label{sec-1-2-3}
Below is a minimal example of the inverse function being partial to $f$:

\begin{align*}
  A &= \{1, 2, 3\} \\
  B &= \{1, 2, 3\} \\
  C &= \{1, 2\} \\
  f(1) &= 1 \\
  f(2) &= 1 \\
  f^{-1}(1) &= 1
\end{align*}
\subsection{Problem 3}
\label{sec-1-3}
Given functions $f$ and $g$ both are from $\mathbb{N}$ to $\mathbb{N}$.
$g$ is known to be onto and for all $n$ in $\mathbb{N}$ it holds that

\begin{equation*}
  (f \circ g)(n) = 2n - 1
\end{equation*}

\begin{enumerate}
\item Prove that $f$ is not surjective.
\item Prove that $f$ is bijective.
\item Show such $f$ and $g$ that satisfy the formula given above.
\end{enumerate}

\subsubsection{Answer 8}
\label{sec-1-3-1}
Suppose, by contradiction, that $f$ was surjective, this would imply
that any even number, in particular, the number 2 in the image of $f$
would have to have its origin in $\mathbb{N}$, but the origin of 2 is in
$\mathbb{Q}$, but not in $\mathbb{N}$, i.e.
$n = \frac{1}{2} \implies 2n - 1 = 2$.  Because we know that $g$ is
surjective, it must be $f$, which is not surjective (the composition
of surjective functions is surjective).
\subsubsection{Answer 9}
\label{sec-1-3-2}
Suppose $f$ is not bijective, this woul3d imply there exist
$x, x' \in \mathbb{N}$ such that $x \neq x'$ while $x = 2n - 1$
and $x' = 2n - 1$, which is a contradiction.
\subsubsection{Answer 10}
\label{sec-1-3-3}
The simplest example would be to take $g$ equal to identity and $f$
being $f(n) = 2n - 1$.  The argument for $f$ not being surjective is
essentially the same as in \ref{sec-1-3-1}.
\subsection{Problem 4}
\label{sec-1-4}
Let $G$ be a group with $*$ being the group operation.  Let also $a \in G$.
Given a function $f : G \to G$ defined as
$\forall x \in G: f(x) = a^{-1} * x * a$.

\begin{enumerate}
\item Prove that $f$ is surjective and bijective.
\item Find $f^{-1}$.
\item Prove that if $b, c \in G$ are each others inverses, then the same is
true of $f(b)$ and $f(c)$.
\end{enumerate}

\subsubsection{Answer 11}
\label{sec-1-4-1}
Suppose $f$ was not bijective, this would imply that there could be
$a^{-1} * x * a = a^{-1} * x' * a$ for some $x \neq x'$.  But using group
cancellation property we could reduce $a^{-1} * x * a = a^{-1} * x' * a$ to
$x = x'$, obtaining contradiction.  Hence $f$ is a bijection.

$x$ is defined to be a member of $G$.  Since $G$ is closed under $*$, it
means that there can't be any values in the domain of $f$ which are not in
its co-domain.  In other words, suppose there was a $x' \in G$, which cannot
be produced by $f(x)$, that is, if we look at the operation table, each
column of this table will have to have as many rows as there are elements.
But no two rows in this column can repeat because this would defy the
cancellation property (i.e. it would mean that for some $p, q, r, s$ it
would be that $p * q = r$ and $p * s = r$, $r$ being the value, which repeats
in the selected column and $q, s$ are the elements for which it repeats.
Hence $f$ must be onto (surjective).
\subsubsection{Answer 12}
\label{sec-1-4-2}
By definition of inverse, applying the group operation to inverse and the
element it is inverse of will give us the identity element:

\begin{align*}
  (a^{-1} * x * a) * (a^{-1} * x * a)^{-1} & = a^{-1} * a
  & \textrm{given} \\
  x * a * (a^{-1} * x * a)^{-1} & = a
  & \textrm{by group cancellation property} \\
  x^{-1} * x * a * (a^{-1} * x * a)^{-1} & = x^{-1} * a
  & \textrm{again, by group cancellation property} \\
  a * (a^{-1} * x * a)^{-1} & = x^{-1} * a
  & \textrm{by group operation on inverses} \\
  a^{-1} * a * (a^{-1} * x * a)^{-1} & = a^{-1} * x^{-1} * a
  & \textrm{by cancellation property} \\
  (a^{-1} * x * a)^{-1} & = a^{-1} * x^{-1} * a
  & \textrm{by group operation on inverses}
\end{align*}

Hence $f^{-1}(x) = a^{-1} * x^{-1} * a$.
\subsubsection{Answer 13}
\label{sec-1-4-3}
Provided $b = c^{-1}, c = b^{-1}$, the next equalities hold:

\begin{equation*}
  \begin{array}{lllllll}
    f(b) & = & a^{-1} * b * a & = & a^{-1} * c^{-1} * a & = & f(c^{-1}) \\
    f(c) & = & a^{-1} * c * a & = & a^{-1} * b^{-1} * a & = & f(b^{-1}) \\
  \end{array}
\end{equation*}

Which is what we were asked to prove.
% Emacs 25.0.50.1 (Org mode 8.2.2)
\end{document}